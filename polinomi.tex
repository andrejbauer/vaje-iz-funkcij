% !TeX root = vaje.tex
\chapter{Polinomi}
\label{cha:polinomi}

\section{Pregled snovi}
\label{sec:polinomi-pregled-snovi}


\subsection{Definicija polinoma}
Polinom je vsaka taka funkcija, ki jo lahko zapišemo v obliki:
\[
p(x)=a_nx^n + a_{n-1}x^{n-1}+ \cdots + a_2x^2 + a_1 x + a_0
\]
Pri tem naravno število n imenujemo stopnja polinoma oz. st(p), koeficienti $ a_j$ so realna števila, koeficient $ a_n$ (tj. tisti, pri najvišji potenci) imenujemo \textbf{vodilni} koeficient, $ a_0$ pa \textbf{prosti člen}.
\subsection{Posebni primeri}
\begin{itemize}
\item Polinom ničte stopnje je konstantni polinom, p(x)=a. V primeru a=0, ga imenujemo ničelni polinom.
\item Polinom prve stopnje je linearna funkcija, p(x)=kx + n
\item Polinom druge stopnje je kvadratna funkcija.
\end{itemize}
\subsection{Računske operacije s polinomi}
Vrednost polinoma v danem številu dobimo tako, da v polinom vstavimo to število. 
Tako kot vse funkcije lahko tudi polinome seštevamo, odštevamo, množimo in delimo (paziti moramo le na deljenje z 0). Veljajo pravila:

Recimo: 
$ p(x)=a_nx^n + \cdots + a_2x^2 + a_1 x + a_0$ 

 in  
$ q(x)=b_nx^n + \cdots + b_2x^2 + b_1 x + b_0$ 

Tedaj velja:
$p(x) + q(x) = (p+q)(x) =(a_n + b_n)x^n +\cdots +(a_2 + b_2)x^2 +( a_1 + b_1) x + (a_0+b_0)$

Podobne formule dobimo tudi za $(p-q)(x), (p*q)(x), (\frac{p}{q})(x)$.
\subsection{Osnovni izrek o deljenju polinomov}
Vsak polinom p(x) (deljenec), lahko delimo s poljubnim neničelnim polinomom q(x) (deljitelj). Zapišemo ju lahko v obliko $\textbf{p(x)=k(x)*q(x) + o(x)}$, pri čemer je $k(x)$ polinom količnik, $o(x)$ pa polinom ostanek. Velja tudi $\textbf{st(o)< st(q)}$. Deljenje polinoma p(x) z polinomom q(x)=x-a, kjer je a neko število, lahko krajše napišemo s Hornerjevim algoritmom.





\section{Vaje}
\label{sec:polinomi-funkcije-vaje}

%%%%%%%%%%%%%%%%%%%%%%%%%%%%%%%%%%%%%%%%%%%%%%%%%%%%%%%%%%%%%%%%%%%%%%
% Odpremo datoteko, v katero se bodo zapisali odgovori za
% to poglavje.

% Določimo ime datoteke, v katero se bodo pisali odgovori.
% Vsako poglavje mora imeti svojo datoteko.
\def\datotekaOdgovori{odgovori-polinomi}

% Odpremo datoteko z odgovori.
\Opensolutionfile{odgovor}[\datotekaOdgovori]

%%%%%%%%%%%%%%%%%%%%%%%%%%%%%%%%%%%%%%%%%%%%%%%%%%%%%%%%%%%%%%%%%%%%%%
% VAJE
%
% Sem vstavimo vaje s pomočjo okolja "vaja". Odgovor napišemo v vajo,
% v okolje "odgovor".



\begin{vaja}
  Poiščite ničle polinoma $x^3 + 3 x + 1$.

  \begin{odgovor}
    Grozna rešitev.
  \end{odgovor}
\end{vaja}

\begin{vaja}
  Še ena vaja.

  \begin{odgovor}
    Rešitev bi bila tu.
  \end{odgovor}
\end{vaja}

%%%%%%%%%%%%%%%%%%%%%%%%%%%%%%%%%%%%%%%%%%%%%%%%%%%%%%%%%%%%%%%%%%%%%%
% Treba je zapredi datoteko z odgovori

\Closesolutionfile{odgovor}

%%%%%%%%%%%%%%%%%%%%%%%%%%%%%%%%%%%%%%%%%%%%%%%%%%%%%%%%%%%%%%%%%%%%%%
% Odgovori

\section{Odgovori}
\label{sec:polinomi-odgovori}

% Vključimo odgovore.
\input{\datotekaOdgovori}


%%% Local Variables:
%%% mode: latex
%%% TeX-master: "vaje"
%%% End:
