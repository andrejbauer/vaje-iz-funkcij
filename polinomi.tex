\chapter{Polinomi}
\label{cha:polinomi}

\section{Pregled snovi}
\label{sec:polinomi-pregled-snovi}


\subsection{Definicija polinoma}
Polinom je vsaka taka funkcija, ki jo lahko zapišemo v obliki:
\[
p(x)=a_nx^n + a_{n-1}x^{n-1}+ \cdots + a_2x^2 + a_1 x + a_0
\]
Pri tem naravno število n imenujemo stopnja polinoma oz. st(p), koeficienti $ a_j$ so realna števila, koeficient $ a_n$ (tj. tisti, pri najvišji potenci) imenujemo vodilni koeficient, $ a_0$ pa \textbf{prosti člen}.
\subsection{Posebni primeri}
\begin{itemize}
\item Polinom ničte stopnje je konstantni polinom, p(x)=a. V primeru a=0, ga imenujemo ničelni polinom.
\item Polinom prve stopnje je linearna funkcija, p(x)=kx + n
\item Polinom druge stopnje je kvadratna funkcija.
\end{itemize}
\subsection{Računske operacije s polinomi}
Vrednost polinoma v danem številu dobimo tako, da v polinom vstavimo to število. Tako kot vse funkcije lahko tudi polinome seštevamo, odštevamo, množimo in delimo (paziti moramo le na deljenje z 0). Veljajo pravila:

Recimo: $ p(x)=a_nx^n + a_{n-1}x^{n-1}+ \cdots + a_2x^2 + a_1 x + a_0$  in   q(x)=b_nx^n + b_{n-1}x^{n-1}+ \cdots + b_2x^2 + b_1 x + b_0$
$p(x) + q(x) = (p+q)(x) =(a_n + b_n)x^n + (a_{n-1}+b_{n-1)}x^{n-1}+ \cdots +(a_2 + b_2)x^2 +( a_1 + b_1) x + (a_0+b_0)$

Podobne formule dobimo tudi za $(p-q)(x), (p*q)(x), (\frac{p}{q})(x)$.
\subsection{Osnovni izrek o deljenju polinomov}
Vsak polinom p(x) (deljenec), lahko delimo s poljubnim neničelnim polinomom q(x) (deljitelj). Zapišemo ju lahko v obliko $p(x)=k(x)*q(x) + o(x)$, pri čemer je $k(x)$ polinom količnik, $o(x)$ pa polinom ostanek. Velja tudi $st(o)< st(q)$. Deljenje polinoma p(x) z polinomom q(x)=x-a, kjer je a neko število, lahko krajše napišemo s Hornerjevim algoritmom.

\subsection{Razcep polinoma, iskanje ničel}
V primeru, ko je število a ničla polinoma p, je ostanek pri deljenju polinoma p z (x-a) enak 0, zato lahko zapišemo $p(x)= k(x)*(x-a)$ postopek ponavljamo, sedaj na k(x). Tako lahko polinom zapišemo v \textbf{ničelno obliko}: 
\[
p(x)= a_n*(x-x_1)(x-x_2)(x-x_3)\cdots(x-x_n)
\]
Pri tem je $a_n$ vodilni koeficient, $x_i$ pa ničle polinoma p(x). V splošnem so te ničle lahko tudi nerealne. v tem primeru nastopajo v konjugiranih parih. Velja tudi st(p)=n. Ta zapis nam omogoča Gaussov izrek: Vsak nekonstanten polinom ima v $\mathbb{C}$ vsaj eno ničlo.

Za iskanje ničel poznamo več metod:
\begin{itemize}
\item Najlažji se zdi \textbf{razcep polinoma}, kjer se držimo pravil za razcep izrazov. Iz te oblike lahko enostavno razberemo ničle, a se metode ne da enostavno uporabiti pri vseh izrazih.
\item Ugibamo možne ničle in jih potem preverimo s \textbf{Hornerjevim algoritmom}. Pri tem upoštevamo dve pravili: cele ničle polinoma, ki ima le cele koeficiente, iščemo le med deljitelji prostega člena ter kandidati za racionalne ničle polinoma s celimi koeficienti so le ulomki, ki imajo v števcu deljitelj prostega člena, v imenovalcu pa deljitelj vodilnega člena.
\item Zgornji metodi nista vedno uporabni, ostaja nam možnost uporabe numeričnih metod (za približke ničel), npr. \textbf{metoda bisekcije}:\begin{enumerate}
\item Najprej poiščemo interval [a, b] na katerem polinom spremeni predznak (tj. da sta vrednosti polinoma v krajiščih različno predznačeni).
\item Izračunamo razpolovišče intervala: $c=\frac{a+b}{2}$
\item Izračunamo p(c). Če je vrednost enaka 0, smo dobili ničlo in postopek je uspešno zaključen. Če dobimo neničelno vrednost, moramo ugotoviti, na katerem od intervalov [a, c] ali [c, b] polinom spremeni predznak. Postopek ponovimo na tem intervalu. Večkrat kot ponovimo postopek, bolj natančen bo naš približek za ničlo. 
\end{enumerate}
\end{itemize}

\subsection{Graf polinoma}
Za risanje grafa moramo prej izračunati nekaj podatkov:
\begin{itemize}
\item Ničle polinoma, so presečišča grafa z abscisno osjo. Najdemo jih po zgoraj opisanih postopkih. Pri tem upoštevamo, da je pomembna tudi stopnja ničel. 
\begin{itemize}
\item V \textbf{enostavnih} ničlah (ničle prve stopnje) graf seka absciso pod nekim kotom.
\item V ničlah \textbf{sode} stopnje se graf abscise le dotakne, v tisti točki je tangent na graf enaka abscisi.
\item V ničlah \textbf{lihe} stopnje, večje od 1, graf seka absciso, a se ji v ničli lepo prilega in je vodoraven. Graf ima v tej ničli vodoravni prevoj.
\end{itemize}
\item \textbf{Začetna vrednost} je presečišče grafa z ordinatno osjo. Izračunamo jo kot p(0) in opazimo, da je enaka prostemu členu.
\item Zavedati se moramo, da je polinom zvezna funkcija, to pomeni, da je njen graf napretrgana krivulja. 
\item Ko gre $x\to \pm \infty$  je graf polinoma podoben grafu vodilnega člena, torej $a_nx^n$:
\begin{enumerate}
\item Če je n liho število, se predznak grafa zamenja v neskončnosti. Če je $a_ n$ pozitiven, graf narašča, če je negativen, pa pada.
\item Če je n sodo število, se predznak grafa ohrani. Če je $a_ n$ pozitiven, se graf prične in konča v pozitivni smeri, če pa je negativen, se prične in konča v $-\infty$.
\end{enumerate}
\item Za podrobnejši graf moramo pogledati tudi odvod polinoma p'(x). Stacionarne točke so vsi x, ki rešijo enačbo $p'(x)=0$. Če odvod v x spremeni predznak iz pozitivnega na negativni, ima graf v x lokalni maksimum, v obratnem primeru pa lokalni minimum. Če predznaka ne spremeni, garf v x nima ekstrema. Velja namreč, da na območju, kjer $p'(x) >0$ graf narašča, kjer $p'(x)<0$ pa graf pada.
\end{itemize}



\section{Vaje}
\label{sec:polinomi-funkcije-vaje}

%%%%%%%%%%%%%%%%%%%%%%%%%%%%%%%%%%%%%%%%%%%%%%%%%%%%%%%%%%%%%%%%%%%%%%
% Odpremo datoteko, v katero se bodo zapisali odgovori za
% to poglavje.

% Določimo ime datoteke, v katero se bodo pisali odgovori.
% Vsako poglavje mora imeti svojo datoteko.
\def\datotekaOdgovori{odgovori-polinomi}

% Odpremo datoteko z odgovori.
\Opensolutionfile{odgovor}[\datotekaOdgovori]

%%%%%%%%%%%%%%%%%%%%%%%%%%%%%%%%%%%%%%%%%%%%%%%%%%%%%%%%%%%%%%%%%%%%%%
% VAJE
%
% Sem vstavimo vaje s pomočjo okolja "vaja". Odgovor napišemo v vajo,
% v okolje "odgovor".

\begin{vaja}
  Poiščite ničle polinoma $x^3 + 3 x + 1$.

  \begin{odgovor}
    Grozna rešitev.
  \end{odgovor}
\end{vaja}

\begin{vaja}
  Še ena vaja.

  \begin{odgovor}
    Rešitev bi bila tu.
  \end{odgovor}
\end{vaja}

%%%%%%%%%%%%%%%%%%%%%%%%%%%%%%%%%%%%%%%%%%%%%%%%%%%%%%%%%%%%%%%%%%%%%%
% Treba je zapredi datoteko z odgovori

\Closesolutionfile{odgovor}

%%%%%%%%%%%%%%%%%%%%%%%%%%%%%%%%%%%%%%%%%%%%%%%%%%%%%%%%%%%%%%%%%%%%%%
% Odgovori

\section{Odgovori}
\label{sec:polinomi-odgovori}

% Vključimo odgovore.
\input{\datotekaOdgovori}


%%% Local Variables:
%%% mode: latex
%%% TeX-master: "vaje"
%%% End:
