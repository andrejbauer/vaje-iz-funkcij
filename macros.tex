% !TeX root = vaje.tex
%%%%%%%%%%%%%%%%%%%%%%%%%%%%%%%%%%%%%%%%%%%%%%%%%%%%%%%%%%%%%%%%%%%%%%
% Množice

% Makro za množice je \set.
% Podamo mu lahko en izbirni argument v oglatih oklepajih []
% in enega ali dva obvezna argumenta v zavitih oklepajih {}.
% Izbirni argument je velikost zavitih oklepajev v zapisu množice.
% Dan je kot število od 0 do 4.
% Če ga ne podamo, se velikost zavitih oklepajev samodejno prilagodi vsebini.
% Če podamo samo en obvezni argument, se množica zapiše kot zaporedje elementov v zavitih oklepajih.
% Če podamo dva obvezna argumenta, se ta dva izpišeta, ločena z navpično črto in obdana z zavitimi oklepaji.
% Primer:
% \set{1, 2, 3}  izpiše  {1, 2, 3}.
% \set{x \in \RR}{x \geq 0}  izpiše  {x ∈ ℝ | x ≥ 0}.

\newcommand{\sizedescriptor}[2]
{
\ifthenelse{\equal{#1}{0}}{}{
\ifthenelse{\equal{#1}{1}}{\big}{
\ifthenelse{\equal{#1}{2}}{\Big}{
\ifthenelse{\equal{#1}{3}}{\bigg}{
\ifthenelse{\equal{#1}{4}}{\Bigg}{
#2}}}}}
}

\NewDocumentCommand{\set}
{O{auto} m G{\empty}}
{\sizedescriptor{#1}{\left}\{ {#2} \ifthenelse{\equal{#3}{}}{}{ \; \sizedescriptor{#1}{\middle}| \; {#3}} \sizedescriptor{#1}{\right}\}}


%%%%%%%%%%%%%%%%%%%%%%%%%%%%%%%%%%%%%%%%%%%%%%%%%%%%%%%%%%%%%%%%%%%%%%
% Številske množice

\newcommand{\NN}{\mathbb{N}}     % naravna števila
\newcommand{\NNz}{\mathbb{N}_0}  % naravna števila z ničlo
\newcommand{\ZZ}{\mathbb{Z}}     % cela števila
\newcommand{\QQ}{\mathbb{Q}}     % racionalna števila
\newcommand{\RR}{\mathbb{R}}     % realna števila

% ukazi za kotne funkcije
\newcommand{\tg}{\operatorname{tg}}
\newcommand{\ctg}{\operatorname{ctg}}
\newcommand{\arctg}{\operatorname{arctg}}

\newcommand{\koordinate}[4]
{
\draw[->] (-#1,0) -- (#2,0) node[below] {$x$};
%\foreach \x in {*,...,-1,1,2,...,*}
%\draw[shift={(\x,0)}] (0pt,2pt) -- (0pt,-2pt) node[below] {\footnotesize $\x$};
\draw[->] (0,-#3) -- (0,#4) node[left] {$y$};
%\foreach \y in {*,...,-1,1,2,...,*}
%\draw[shift={(0,\y)}] (2pt,0pt) -- (-2pt,0pt) node[left] {\footnotesize $\y$};
\node[below left] at (0,0) {\footnotesize $0$};
}
%argument ena in dva se navezujeta na x os, 3 in 4 pa na y os; argumenti so samo celoštevilski. Kar je zakomentirano naj vsak kopira in vstavi svoje meje namesto zvezdic (pazi, da ne prideš do roba). Uporabljamo barve: blue, red, green, orange, grey (siva je za asimptote). Vsi originalni grafi so modre barve, vsak naslednji pa sledi vrstnemu redu prej naštetih. Debelina črt: debelina grafa je 2pt, vse pomožne črte 1pt, vse točke enake debeline 1,5pt. 

%%% Local Variables:
%%% mode: latex
%%% TeX-master: "vaje"
%%% End:
