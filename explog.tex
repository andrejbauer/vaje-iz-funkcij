% !Tex root = vaje.tex
\chapter{Eksponentna in logaritemska funkcija}
\label{cha:exp-log}

\section{Pregled snovi}
\label{sec:exp-log-pregled-snovi}

Pregled snovi.

\section{Vaje}
\label{sec:exp-log-vaje}

%%%%%%%%%%%%%%%%%%%%%%%%%%%%%%%%%%%%%%%%%%%%%%%%%%%%%%%%%%%%%%%%%%%%%%
% Odpremo datoteko, v katero se bodo zapisali odgovori za
% to poglavje.

% Določimo ime datoteke, v katero se bodo pisali odgovori.
% Vsako poglavje mora imeti svojo datoteko.
\def\datotekaOdgovori{odgovori-explog}

% Odpremo datoteko z odgovori.
\Opensolutionfile{odgovor}[\datotekaOdgovori]

%%%%%%%%%%%%%%%%%%%%%%%%%%%%%%%%%%%%%%%%%%%%%%%%%%%%%%%%%%%%%%%%%%%%%%
% VAJE
%
% Sem vstavimo vaje s pomočjo okolja "vaja". Odgovor napišemo v vajo,
% v okolje "odgovor".

\begin{vaja}
  Izračunajte $2^4$.

  \begin{odgovor}
    $2^4 = 4^2$.
  \end{odgovor}
\end{vaja}

\begin{vaja}
  Še ena vaja.

  \begin{odgovor}
    Rešitev bi bila tu.
  \end{odgovor}
\end{vaja}

%%%%%%%%%%%%%%%%%%%%%%%%%%%%%%%%%%%%%%%%%%%%%%%%%%%%%%%%%%%%%%%%%%%%%%
% Treba je zapredi datoteko z odgovori

\Closesolutionfile{odgovor}

%%%%%%%%%%%%%%%%%%%%%%%%%%%%%%%%%%%%%%%%%%%%%%%%%%%%%%%%%%%%%%%%%%%%%%
% Odgovori

\section{Odgovori}
\label{sec:explog-odgovori}

% Vključimo odgovore.
\input{\datotekaOdgovori}


%%% Local Variables:
%%% mode: latex
%%% TeX-master: "vaje"
%%% End:
