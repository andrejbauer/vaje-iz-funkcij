% !Tex root = vaje.tex
\chapter{Eksponentna in logaritemska funkcija}
\label{cha:exp-log}

\section{Pregled snovi}
\label{sec:exp-log-pregled-snovi}

Pregled snovi.

\section{Vaje}
\label{sec:exp-log-vaje}

%%%%%%%%%%%%%%%%%%%%%%%%%%%%%%%%%%%%%%%%%%%%%%%%%%%%%%%%%%%%%%%%%%%%%%
% Odpremo datoteko, v katero se bodo zapisali odgovori za
% to poglavje.

% Določimo ime datoteke, v katero se bodo pisali odgovori.
% Vsako poglavje mora imeti svojo datoteko.
\def\datotekaOdgovori{odgovori-explog}

% Odpremo datoteko z odgovori.
\Opensolutionfile{odgovor}[\datotekaOdgovori]

%%%%%%%%%%%%%%%%%%%%%%%%%%%%%%%%%%%%%%%%%%%%%%%%%%%%%%%%%%%%%%%%%%%%%%
% VAJE
%
% Sem vstavimo vaje s pomočjo okolja "vaja". Odgovor napišemo v vajo,
% v okolje "odgovor".

\begin{vaja}
 Poenostavi naslednji izraz \(\sqrt{x^{\frac{10}{3}}y^{\frac{4}{2}}}(x^{2}y^{\frac{1}{2}})^{4}\)

  \begin{odgovor}
   Vsa števila moramo najprej napisati v taki obliki da so potence realna števila. 
	$$\sqrt{x^{\frac{10}{3}}y^{\frac{4}{2}}}(x^{2}y^{\frac{1}{2}})^{4}=(x^{\frac{10}{3}}y^{\frac{4}{2}})^{\frac{1}{2}}(x^{2}y^{\frac{1}{2}})^{4}=x^{\frac{10}{6}}yx^{8}y^{2}=y^{3}x^{\frac{29}{3}}$$
  \end{odgovor}
\end{vaja}

\begin{vaja}
  Čim bolj poenostavi naslednji izraz \(\,x^{\frac{5}{3}}y^{\frac{8}{7}}\sqrt{\sqrt{x^{\frac{1}{7}}y^{\frac{14}{3}}}}\)

  \begin{odgovor}
   Rešitev : $x^{\frac{1}{28}} y^{\frac{97}{42}}$
  \end{odgovor}
\end{vaja}

\begin{vaja}
  napiši vse rešitve naslednje enačbe \( 8^m=32\)

  \begin{odgovor}
   Enačbo želimo najprej  preorediti v obliko $a^{f(x)}=b^{g(x)}$ potem pa imamo 2 različni možnosti:
	\begin{itemize}
		\item[a)] $a=b$ 

			v tem primeru velja $a^{f(x)}=a^{g(x)}$ kar pa drži natanko tedaj ko je $f(x)=g(x)$ saj je $a^x$ injektivna
			
		\item[b)] $a\neq b$
	\end{itemize}
  \end{odgovor}
\end{vaja}

\begin{vaja}
  Še ena vaja.

  \begin{odgovor}
    Rešitev bi bila tu.
  \end{odgovor}
\end{vaja}

%%%%%%%%%%%%%%%%%%%%%%%%%%%%%%%%%%%%%%%%%%%%%%%%%%%%%%%%%%%%%%%%%%%%%%
% Treba je zapredi datoteko z odgovori

\Closesolutionfile{odgovor}

%%%%%%%%%%%%%%%%%%%%%%%%%%%%%%%%%%%%%%%%%%%%%%%%%%%%%%%%%%%%%%%%%%%%%%
% Odgovori

\section{Odgovori}
\label{sec:explog-odgovori}

% Vključimo odgovore.
\input{\datotekaOdgovori}


%%% Local Variables:
%%% mode: latex
%%% TeX-master: "vaje"
%%% End:
