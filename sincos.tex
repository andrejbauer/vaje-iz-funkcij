% !TeX root = vaje.tex

\chapter{Kotne funkcije}
\label{cha:sin-cos}

\section{Pregled snovi}
\label{sec:sin-cos-pregled-snovi}
\subsection{Definicije kotnih funkcij v pravokotnem trikotniku}
Funkcije sinus, kosinus, tangens in kotagens imenujemo kotne oziroma trigonometrične funkcije. V nadaljevanju jih bomo označevali z $\sin x$, $\cos x$, $\tg x$ in $\ctg x$. V pravokotnem trikotniku jih definiramo z razmerji stranic in so odvisne od velikosti kota. 


\begin{tikzpicture}[line cap=round, line join=round,>=triangle 45,x=1.0cm,y=1.0cm]
\draw[line width=.pt,color=blue,fill=blue,fill opacity=0.10000000149011612] (1.152776418235458,3.908501496011466) -- (1.3842749222239916,3.641277914246924) -- (1.6514985039885335,3.872776418235458) -- (1.42,4.14) -- cycle; 
\draw [shift={(-2.32,0.9)},line width=1.pt,color=blue,fill=blue,fill opacity=0.10000000149011612] (0,0) -- (0.01628264631072814:1.2) arc (0.01628264631072814:40.902716394791724:1.2) -- cycle;
\draw [line width=2.pt] (-2.32,0.9)-- (4.2252335288255765,0.9018600624050446);
\draw [line width=2.pt] (4.2252335288255765,0.9018600624050446)-- (1.42,4.14);
\draw [line width=2.pt] (1.42,4.14)-- (-2.32,0.9);
\draw (-2.44,1.25) node {$A$};
\draw (1.56,4.51) node {$C$};
\draw (4.36,1.27) node {$B$};
\draw (1.18,0.40) node {\text{hipotenuza}};
\draw (4.54,3.06) node {\text{nasprotna kateta}};
\draw (-1.9,3.06) node {\text{priležna kateta}};
\draw[color=blue] (-0.66,1.47) node {$\alpha$};
\end{tikzpicture}

\begin{align*}
\sin \alpha &= \frac{\text{nasprotna kateta}}{\text{hipotenuza}} \\
\cos \alpha &= \frac{\text{priležna kateta}}{\text{hipotenuza}}  \\
\tg \alpha &= \frac{\text{nasprotna kateta}}{\text{priležna kateta}}\\
\ctg \alpha &= \frac{\text{priležna kateta}}{\text{nasprotna kateta}}
\end{align*}

\begin{zgled}
V pravokotnem trikotniku s kotom $\gamma= 90^{\circ}$, merijo stranice $a=9$, $b=12$ in $c=15$. Izračunaj vrednosti $\cos\alpha$, $\sin\alpha$, $\tg\alpha$, $\tg\beta$ na štiri decimalna mesta natančno. Izračunaj še kota $\alpha$ in $\beta$ na dve decimalni mesti natančno.
\end{zgled}

\begin{resitev}
Glede na kot $\alpha$ je $b$ priležna kateta, $a$ nasprotna kateta in $c$ hipotenuza. Po definiciji:
\begin{equation*}
\cos\alpha= \frac{b}{c}= 0,8000 \quad \quad
\sin\alpha = \frac{a}{c}= 0,6000 \quad \quad
\tg \alpha= \frac{a}{b}= 0,7500
\end{equation*}
Glede na kot $\beta$ je $a$ priležna kateta, $b$ nasprotna kateta in $c$ hipotenuza. Po definiciji:
\begin{equation*}
\tg\beta= \frac{b}{a}= 1,3333 
\end{equation*}
Izračunajmo še kota. Ugotovili smo že, da velja $\cos\alpha= \frac{b}{c}$. Iz tega sledi:
\begin{equation*}
\alpha= \arccos \left(\frac{b}{c}\right) = 36,87^{\circ}
\end{equation*}
Enako še za $\beta$:
\begin{equation*}
\beta= \arctg \left(\frac{b}{a}\right) = 53,13^{\circ}
\end{equation*}
Pomoč: za izračun $\arccos x$ v kalkuator vtipkajte $\cos^{-1}x$.
\end{resitev}

\subsection{Zveze med kotnimi funkcijami}

Naslednje enačbe uporabljamo pri poenostavljanju in reševanju trigonometričnih enačb. Funkciji tangens in kotangens lahko definiramo s sinusom in kosinusom:
\begin{align*}
\tg \alpha &= \frac{\sin \alpha}{\cos \alpha}, &
\ctg \alpha &= \frac{\cos \alpha}{\sin \alpha} \quad &
\Rightarrow \ctg \alpha \cdot \tg \alpha = 1
\end{align*}
Funkciji sinus in kosinus pa povezuje enačba
\begin{equation*}
\sin^2 \alpha + \cos^2 \alpha = 1.
\end{equation*}
Če enačbo delimo s $\cos^2\alpha$ in s $\sin^2\alpha$  pa dobimo povezavi
\begin{equation*}
1+ \tg^2 \alpha = \frac{1}{\cos^2\alpha}\quad \text{in} \quad
1+ \ctg^2 \alpha= \frac{1}{\sin^2\alpha}.
\end{equation*}
V pravokotnem trikotniku, kjer je $\gamma={90}^{\circ}$, sta kota $\alpha$ in $\beta$ komplementarna. Zato zanju velja $\alpha+\beta= \frac{\pi}{2}$. Tako dobimo zveze:
\begin{align*}
\sin\alpha &= \cos(\frac{\pi}{2}- \alpha) \\
\cos\alpha &= \sin(\frac{\pi}{2}- \alpha) \\
\tg \alpha &= \ctg(\frac{\pi}{2}- \alpha) \\
\ctg\alpha &= \tg(\frac{\pi}{2}- \alpha).
\end{align*}

\subsection{Definicije kotnih funkcij na enotski krožnici}

Kotne funkcije definiramo tudi s pomočjo enotske krožnice. Poltrak iz koordinatnega izhodišča s pozitivnim delom abscisne osi določa kot $\alpha$. Presečišče krožnice in poltraka vzamemo za točko $P$. S pravokotnico na $x$-os skozi $P$ dobimo pravokotni trikotnik s hipotenuzo dolžine 1. Abscisa presečišča $P$ tako predstavlja vrednost funkcije $\cos\alpha$, ordinata pa vrednost funkcije $\sin\alpha$. 

Funkciji tangens in kotangens prav tako definiramo s pomočjo enotske krožnice in poltraka iz izhodišča. Za ponazoritev $\tg\alpha$ narišemo vzporednico $y$-osi skozi $(1,0)$. V dobljenem pravokotnem trikotniku meri kateta, ki leži na $x$-osi, 1. Ordinata presečišča vzporednice in poltraka predstavlja velikost funkcije $\tg\alpha$. Za ponazoritev $\ctg\alpha$ pa narišemo vzporednico $x$-osi skozi (0,1). Abscisa presečišča vzporednice in poltraka predstavlja velikost funkcije $\ctg\alpha$. 

\

\begin{tikzpicture}[line cap=round,line join=round,>=triangle 45,x=3.0cm,y=3.0cm]
\draw[->,color=black] (-1.81936,0.) -- (2.0581866666666713,0.);
\foreach \x in {-1.5,-1.,-0.5,0.5,1.,1.5,2.}
\draw[shift={(\x,0)},color=black] (0pt,-2pt);
\draw[color=black] (1.94896,0.027306666666666635) node [anchor=south west] { $x$};
\draw[->,color=black] (0.,-1.4393955555555527) -- (0.,1.5506844444444439);
\foreach \y in {-1.,-0.5,0.5,1.,1.5}
\draw[shift={(0,\y)},color=black] (-2pt,0pt);
\draw[color=black] (0.034133333333333356,1.4004977777777774) node [anchor=west] { $y$};
\clip(-1.81936,-1.4393955555555527) rectangle (2.0581866666666713,1.5506844444444439);
\foreach \x in {-1,1}
\draw[shift={(\x,0)}] (0pt,2pt) -- (0pt,-2pt) node[below] {\footnotesize $\x$};
\foreach \y in {-1,1}
\draw[shift={(0,\y)}] (2pt,0pt) -- (-2pt,0pt) node[left] {\footnotesize $\y$};
\draw [shift={(0.,0.)},line width=2.pt] (0,0) -- (0.:0.2048) arc (0.:56.44300446165065:0.2048) -- cycle;
\draw [line width=1.pt,color=gray] (0.,0.) circle (3.0cm);
\draw [line width=2.pt,domain=0.0:2.0581866666666713] plot(\x,{(-0.--0.8333363648462626*\x)/0.5527662281876642});
\draw [line width=2.pt,color=blue] (0.5527662281876642,0.8333363648462626)-- (0.5527662281876642,0.);
\draw [line width=2.pt,color=red] (0.5527662281876642,0.)-- (0.,0.);
\draw [fill=gray] (1.,0.) circle (1.5pt);
\draw [fill=gray] (0.,1.) circle (1.5pt);
\draw [fill=gray] (-1.,0.) circle (1.5pt);
\draw [fill=gray] (0.,-1.) circle (1.5pt);
\draw [fill=gray] (0.,0.) circle (1.5pt);
\draw [fill=gray] (0.5527662281876642,0.8333363648462626) circle (1.5pt) node[right] {$P$};
\draw[color=gray] (0.98928,0.84257777777778);
\draw [fill=gray] (0.5527662281876642,0.) circle (1.5pt);
\draw[color=blue] (0.704773333333337,0.4413511111111119) node {$\sin\alpha$};
\draw[color=red] (0.3856533333333367,-0.07699555555555419) node {$\cos\alpha$};
\draw (0.26181333333333658,0.09319111111111231) node {$\alpha$};
\draw [line width=2.pt,color=green] (0.,1.)-- (0.67,1.);
\draw [line width=2.pt,color=orange] (1.,0.)-- (1.,1.5200985915492955);
\draw[color=green] (0.3824,1.1132) node {$\ctg\alpha$};
\draw[color=orange] (1.1748,0.674) node {$\tg\alpha$};
\end{tikzpicture}

Vrednost funkcije sinus in kosinus se ne spremeni, če vrednosti kota prištejemo večkratnik kota $2\pi$. Zato sta periodični funkciji z osnovno periodo $2\pi$. Funkciji tangens in kotangens sta periodični z osnovno periodo $\pi$. Velja:
\begin{align*}
\sin\alpha&=\sin(\alpha+k 2\pi) \\
\cos\alpha&= \cos(\alpha +k2\pi) \\
\tg\alpha&= \tg(\alpha +k\pi)\\
\ctg\alpha&= \ctg(\alpha +k\pi); \quad k \in\mathbb{Z}
\end{align*}

\ 

Kote, ki so večji od $90^{\circ}$ in manjši od $360^{\circ}$, pretvarjamo v ostre kote po naslednjih obrazcih:
\begin{itemize}
\item $90^{\circ} <\alpha<180^{\circ}$ oz. $\frac{\pi}{2}<\alpha<\pi$
\begin{align*}
\sin(180^{\circ}-\alpha)&=\sin\alpha \\
\cos(180^{\circ}-\alpha)&=-\cos\alpha \\
\tg(180^{\circ}-\alpha) &=-\tg\alpha \\
\ctg(180^{\circ}-\alpha)&= -\ctg\alpha
\end{align*}

\item $180^{\circ} <\alpha<270^{\circ}$ oz. $\pi<\alpha<\frac{3\pi}{2}$
\begin{align*}
\sin(180^{\circ}+\alpha)&=-\sin\alpha \\
\cos(180^{\circ}+\alpha)&=-\cos\alpha \\
\tg(180^{\circ}+\alpha) &=\tg\alpha \\
\ctg(180^{\circ}+\alpha)&= \ctg\alpha
\end{align*}

\item $270^{\circ} <\alpha<360^{\circ}$ oz. $\frac{3\pi}{2}<\alpha<2\pi$
\begin{align*}
\sin(360^{\circ}-\alpha)&=-\sin\alpha \\
\cos(360^{\circ}-\alpha)&=\cos\alpha \\
\tg(360^{\circ}-\alpha) &=-\tg\alpha \\
\ctg(3600^{\circ}-\alpha)&=-\ctg\alpha
\end{align*}
\end{itemize}

\subsection{Adicijski izreki}
Adicijske izreke uporabljamo za izračun kotnih funkcij vsote in razlike kotov, pri poenostavljanju izrazov in računanju enačb.
\begin{align*}
\cos(\alpha+ \beta) &= \cos\alpha \cdot \cos\beta - \sin\alpha \cdot \sin\beta \\
\cos(\alpha - \beta) &= \cos\alpha \cdot \cos\beta + \sin\alpha \cdot \sin\beta \\
\sin(\alpha + \beta) &= \sin\alpha \cdot \cos\beta + \cos\alpha \cdot \sin\beta \\
\sin(\alpha - \beta) &= \sin\alpha \cdot \cos\beta - \cos\alpha \cdot \sin\beta \\
\tg(\alpha + \beta) &= \frac{\tg\alpha + \tg\beta}{1- \tg\alpha \cdot \tg\beta} \\
\tg(\alpha - \beta) &= \frac{\tg\alpha - \tg\beta}{1+ \tg\alpha \cdot \tg\beta}
\end{align*}
Iz zgornjih enačb sledijo enačbe kotnih funkcij dvojnega kota:
\begin{align*}
\cos(2\alpha) &= \cos^2\alpha- \sin^2\alpha\\
\sin(2\alpha) &= 2\sin\alpha \cos\alpha \\
\tg(2\alpha) &= \frac{2\tg\alpha}{1- \tg^2\alpha} \\
\end{align*}

\subsection{Trigonometrične enačbe}

Trigonometrične enačbe so enačbe, v katerih nastopa neznanka kot argument katere od kotnih funkcij. Začnimo s preprostimi enačbami, v katerih nastopa le ena kotna funkcija.

\begin{itemize}
\item $\sin x=a \mid |a|\le 1$

Rešitev enačbe so presečišča grafov $y=\sin x$ in $y=a$. Upoštevati moramo,da je $\sin x$ periodična funkcija s periodo $2\pi$ in da funcija doseže vrednost $a$ pri dveh kotih. Rešitve so tako:
\begin{align*}
x_1&= \arcsin a + 2k\pi, k\in \mathbb{Z} \\
x_2&= \pi -\arcsin a + 2k\pi, k\in \mathbb{Z}
\end{align*}

\item $\cos x=a \mid |a|\le 1$

Rešitve enačbe so presečišča grafov $y=\cos x$ in $y=a$. Upoštevati moramo periodičnost in dva različna kota, kjer ima funkcija kosinus vrednost $a$. Rešitve so tako:
\begin{align*}
x_1&= \arccos a + 2k\pi, k\in \mathbb{Z} \\
x_2&= -\arccos a + 2k\pi, k\in \mathbb{Z}
\end{align*}

\item $\tg x =a$

Rešitve enačbe so presečišča grafov $y=\tg x$ in $y=a$. Funkcija tangens ima periodo $\pi$. Rešitve so tako:
\begin{equation*}
x= \arctg a +k\pi, k \in \mathbb{Z}
\end{equation*}
\end{itemize}

Kadar v enačbi nastopajo različne kotne funkcije, poskušamo vse izraziti le z eno. Pomagamo si z zvezami med kotnimi funkcijami in adicijskimi izreki. Če je enačba sestavljena samo iz členov $\sin^2x$, $\cos^2x$, $\sin\cdot\cos$, $1$(vsi členi 2. stopnje), lahko enačbo delimo s $\cos^2x$ in rešujemo kvadratno enačbo za $\tg x$. Za lažje reševanje lahko to uvedemo kot novo neznanko. 

\subsection{Grafi kotnih funkcij}
Kot smo spoznali, so kotne funkcije nekaj osnovnega za razumevanja tako preproste kot osnovne geometrije. Prav tako so nekaj posebnega tudi grafi kotnih funkcij, katerih periodičnost je iz njih lepo razvidna. \\

Graf, ki predstavlja funkcijo $f(x) = sinx$, imenujemo sinusoida. Pri risanju vedno privzamemo, da je argument $x$ kot, izražen v radianih.

% graf sinusa v osnovni obliki
\definecolor{ffqqqq}{rgb}{1.,0.,0.}
\definecolor{qqwuqq}{rgb}{0.,0.6,0.}
\definecolor{cqcqcq}{rgb}{0.75,0.75,0.75}
\begin{tikzpicture}[line cap=round,line join=round,>=triangle 45,x=1.0cm,y=1.0cm]
\draw [color=cqcqcq,, xstep=1.6cm,ystep=1.0cm] (-3.5,-2.5) grid (7.5,2.5);
\draw[->,color=black] (-3.5,0.) -- (7.5,0.);
\draw[shift={(-3.14,0)},color=black] (0pt,2pt) -- (0pt,-2pt) node[below] {\footnotesize $-\pi$};
\draw[shift={(-1.57,0)},color=black] (0pt,2pt) -- (0pt,-2pt) node[below] {\footnotesize $-\pi/2$};
\draw[shift={(1.57,0)},color=black] (0pt,2pt) -- (0pt,-2pt) node[below] {\footnotesize $\pi/2$};
\draw[shift={(3.14,0)},color=black] (0pt,2pt) -- (0pt,-2pt) node[below] {\footnotesize $\pi$};
\draw[shift={(4.71,0)},color=black] (0pt,2pt) -- (0pt,-2pt) node[below] {\footnotesize $3\pi/2$};
\draw[shift={(6.28,0)},color=black] (0pt,2pt) -- (0pt,-2pt) node[below] {\footnotesize $2\pi$};
\draw[->,color=black] (0.,-2.5) -- (0.,2.5);
\foreach \y in {-2.,-1.,1.,2.}
\draw[shift={(0,\y)},color=black] (2pt,0pt) -- (-2pt,0pt) node[left] {\footnotesize $\y$};
\draw[color=black] (0pt,-10pt) node[right] {\footnotesize $0$};
\clip(-3.5,-2.5) rectangle (7.5,2.5);
\draw[line width=1.3pt,color=qqwuqq,smooth,samples=100,domain=-3.5:7.5] plot(\x,{sin(((\x))*180/pi)});
\draw [line width=1.pt,dash pattern=on 2pt off 2pt,color=ffqqqq,domain=-3.5:7.5] plot(\x,{(--1.5707963267948966-0.*\x)/1.5707963267948966});
\draw [line width=1.pt,dash pattern=on 2pt off 2pt,color=ffqqqq,domain=-3.5:7.5] plot(\x,{(-4.40-0.*\x)/4.40});
\end{tikzpicture}

Opazimo, da je funkcija sinus periodična funkcija s periodo $2\pi$, ter da je to liha funkcija, torej
\[
sin(-x) = -sin(x)
\]
Sinus je funkcija, ki je definirana na celi realni osi, zavzame pa le vrednosti na intervalu $[-1, 1]$. Kot je razvidno z grafa, so ničle funkcije vsi večkratniki števila $\pi$ in sicer 
\[
x = k\pi, \:   k\in \mathbb{Z}.
\]

Funkcija periodično zavzame tudi maksimume in minimume. V točkah, kjer je 
\[
x = \frac{\pi}{2} + 2k\pi \: ; \: k \in \mathbb{Z},
\]
doseže svoje maksimume ($y = 1$), v točkah pa, kjer je  
\[
x = \frac{3\pi}{2} + 2k\pi \: ; \: k \in \mathbb{Z},
\] 
doseže svoje minimume ($y = -1$).

Očitno je, da je funkcija sinus zvezna.\\

Funkcija $f(x) = cosx$ si je s sinusom zelo podobna.

% graf kosinusa v osnovni obliki
\definecolor{ffqqqq}{rgb}{1.,0.,0.}
\definecolor{qqwuqq}{rgb}{0.,0.6,0.}
\definecolor{cqcqcq}{rgb}{0.75,0.75,0.75}
\begin{tikzpicture}[line cap=round,line join=round,>=triangle 45,x=1.0cm,y=1.0cm]
\draw [color=cqcqcq,, xstep=1.57cm,ystep=1.0cm] (-3.5,-2.5) grid (7.5,2.5);
\draw[->,color=black] (-3.5,0.) -- (7.5,0.);
\draw[shift={(-3.14,0)},color=black] (0pt,2pt) -- (0pt,-2pt) node[below] {\footnotesize $-\pi$};
\draw[shift={(-1.57,0)},color=black] (0pt,2pt) -- (0pt,-2pt) node[below] {\footnotesize $-\pi/2$};
\draw[shift={(1.57,0)},color=black] (0pt,2pt) -- (0pt,-2pt) node[below] {\footnotesize $\pi/2$};
\draw[shift={(3.14,0)},color=black] (0pt,2pt) -- (0pt,-2pt) node[below] {\footnotesize $\pi$};
\draw[shift={(4.71,0)},color=black] (0pt,2pt) -- (0pt,-2pt) node[below] {\footnotesize $3\pi/2$};
\draw[shift={(6.28,0)},color=black] (0pt,2pt) -- (0pt,-2pt) node[below] {\footnotesize $2\pi$};
\draw[->,color=black] (0.,-2.5) -- (0.,2.5);
\foreach \y in {-2.,-1.,1.,2.}
\draw[shift={(0,\y)},color=black] (2pt,0pt) -- (-2pt,0pt) node[left] {\footnotesize $\y$};
\draw[color=black] (0pt,-10pt) node[right] {\footnotesize $0$};
\clip(-3.5,-2.5) rectangle (7.5,2.5);
\draw[line width=1.3pt,color=qqwuqq,smooth,samples=100,domain=-3.5:7.5] plot(\x,{cos(((\x))*180/pi)});
\draw [line width=1.pt,dash pattern=on 2pt off 2pt,color=ffqqqq,domain=-3.5:7.5] plot(\x,{(--6.283185307179586-0.*\x)/6.283185307179586});
\draw [line width=0.8pt,dash pattern=on 2pt off 2pt,color=ffqqqq,domain=-3.5:7.5] plot(\x,{(-3.141592653589793-0.*\x)/3.141592653589793});
\end{tikzpicture}

Razlikujeta se pravzaprav le v tem, da je, če opazujemo grafa funkcij, graf kosinusa za $\frac{\pi}{2}$ zamaknjen graf sinusa v levo. Tokrat je funkcija kosinus soda, kar pomeni, da velja
\[
cos(-x) = cosx
\]
Definicijsko območje in zaloga vrednosti sta enaka kot pri sinusu, se pa kosinus razlikuje v ničlah in ekstremih. Funkcija kosinus ima ničle v točkah 
\[
x = \frac{\pi}{2} + k\pi \: ; \: k \in \mathbb{Z},
\]
maksimume zavzame pri
\[
x =2\pi + 2 k\pi, \:   k\in \mathbb{Z}
\]
in minimume pri
\[
x =\pi + 2 k\pi, \:   k\in \mathbb{Z}.
\]

Funkcija $f(x) = \tg x$ je, kot smo že prej spoznali, pravzaprav funkcija razmerja med sinusom in kosinusom:
\[
f(x) = tanx = \frac{sinx}{cosx}.
\]
Tangens torej zavzame vrednost $f(x) = 0$ v ničlah sinusa, v ničlah kosinusa pa ima graf tangensa pole. 
Definicijsko območje funkcije je torej unija intervalov ($\frac{\pi}{2} + k\pi$, $\frac{3\pi}{2} + k\pi$), pri čemer je $k \in \mathbb{Z}$. \\

% graf tangensa v osnovni obliki
\definecolor{ffqqqq}{rgb}{1.,0.,0.}
\definecolor{qqwuqq}{rgb}{0.,0.6,0.}
\definecolor{cqcqcq}{rgb}{0.75,0.75,0.75}
\begin{tikzpicture}[line cap=round,line join=round,>=triangle 45,x=1.0cm,y=1.0cm]
\draw [color=cqcqcq,, xstep=1.57cm,ystep=1.0cm] (-4.5,-4.3) grid (4.5,4.3);
\draw[->,color=black] (-4.7,0.) -- (4.7,0.);
\draw[shift={(-3.14,0)},color=black] (0pt,2pt) -- (0pt,-2pt) node[below] {\footnotesize $-\pi$};
\draw[shift={(-1.57,0)},color=black] (0pt,2pt) -- (0pt,-2pt) node[below] {\footnotesize $-\pi/2$};
\draw[shift={(1.57,0)},color=black] (0pt,2pt) -- (0pt,-2pt) node[below] {\footnotesize $\pi/2$};
\draw[shift={(3.14,0)},color=black] (0pt,2pt) -- (0pt,-2pt) node[below] {\footnotesize $\pi$};
\draw[->,color=black] (0.,-4.6) -- (0.,4.6);
\foreach \y in {-4.,-3.,-2.,-1.,1.,2.,3.,4.}
\draw[shift={(0,\y)},color=black] (2pt,0pt) -- (-2pt,0pt) node[left] {\footnotesize $\y$};
\draw[color=black] (0pt,-10pt) node[right] {\footnotesize $0$};
\clip(-4.5,-4.3) rectangle (4.5,4.3);
\draw[line width=1.3pt,color=qqwuqq,smooth,samples=100,domain=-4.5:-1.58] plot(\x,{tan(((\x))*180/pi)});
\draw[line width=1.3pt,color=qqwuqq,smooth,samples=100,domain=-1.56:1.56] plot(\x,{tan(((\x))*180/pi)});
\draw[line width=1.3pt,color=qqwuqq,smooth,samples=100,domain=1.58:4.5] plot(\x,{tan(((\x))*180/pi)});
\draw [line width=1.pt,dash pattern=on 2pt off 2pt,color=ffqqqq] (-1.57,-4.3) -- (-1.57,4.3);
\draw [line width=1.pt,dash pattern=on 2pt off 2pt,color=ffqqqq] (1.57,-4.3) -- (1.57,4.3);
\end{tikzpicture}

Od kotnih funkcij nam ostane le še funkcija $f(x) = \ctg x$, ki pa je pravzaprav inverz funkcije tangensa. Ničle ima tam, kjer ima tangens pole, in pole tam, kjer ima tangens pole.
Graf funkcije je torej takšen: \\

% graf kotangensa v osnovni obliki
\definecolor{ffqqqq}{rgb}{1.,0.,0.}
\definecolor{qqwuqq}{rgb}{0.,0.6,0.}
\definecolor{cqcqcq}{rgb}{0.75,0.75,0.75}
\begin{tikzpicture}[line cap=round,line join=round,>=triangle 45,x=1.0cm,y=1.0cm]
\draw [color=cqcqcq,, xstep=1.57cm,ystep=1.0cm] (-4.5,-4.3) grid (4.5,4.3);
\draw[->,color=black] (-4.7,0.) -- (4.7,0.);
\draw[shift={(-3.14,0)},color=black] (0pt,2pt) -- (0pt,-2pt) node[below] {\footnotesize $-\pi$};
\draw[shift={(-1.57,0)},color=black] (0pt,2pt) -- (0pt,-2pt) node[below] {\footnotesize $-\pi/2$};
\draw[shift={(1.57,0)},color=black] (0pt,2pt) -- (0pt,-2pt) node[below] {\footnotesize $\pi/2$};
\draw[shift={(3.14,0)},color=black] (0pt,2pt) -- (0pt,-2pt) node[below] {\footnotesize $\pi$};
\draw[->,color=black] (0.,-4.6) -- (0.,4.6);
\foreach \y in {-4.,-3.,-2.,-1.,1.,2.,3.,4.}
\draw[shift={(0,\y)},color=black] (2pt,0pt) -- (-2pt,0pt) node[left] {\footnotesize $\y$};
\draw[color=black] (0pt,-10pt) node[right] {\footnotesize $0$};
\clip(-4.5,-4.3) rectangle (4.5,4.3);
\draw[line width=1.3pt,color=qqwuqq,smooth,samples=100,domain=-4.5:-3.15] plot(\x,{cot(((\x))*180/pi)});
\draw[line width=1.3pt,color=qqwuqq,smooth,samples=100,domain=-3.13:-0.01] plot(\x,{cot(((\x))*180/pi)});
\draw[line width=1.3pt,color=qqwuqq,smooth,samples=100,domain=0.01:3.13] plot(\x,{cot(((\x))*180/pi)});
\draw[line width=1.3pt,color=qqwuqq,smooth,samples=100,domain=3.15:4.5] plot(\x,{cot(((\x))*180/pi)});
\draw [line width=1.2pt,dash pattern=on 2pt off 2pt,color=ffqqqq] (-3.14,-4.3) -- (-3.14,4.3);
\draw [line width=1.2pt,dash pattern=on 2pt off 2pt,color=ffqqqq] (0.,-4.3) -- (0.,4.3);
\draw [line width=1.2pt,dash pattern=on 2pt off 2pt,color=ffqqqq] (3.14,-4.3) -- (3.14,4.3);
\end{tikzpicture}


\section{Vaje}
\label{sec:sin-cos-vaje}

%%%%%%%%%%%%%%%%%%%%%%%%%%%%%%%%%%%%%%%%%%%%%%%%%%%%%%%%%%%%%%%%%%%%%%
% Odpremo datoteko, v katero se bodo zapisali odgovori za
% to poglavje.

% Določimo ime datoteke, v katero se bodo pisali odgovori.
% Vsako poglavje mora imeti svojo datoteko.
\def\datotekaOdgovori{odgovori-sincos}

% Odpremo datoteko z odgovori.
\Opensolutionfile{odgovor}[\datotekaOdgovori]

%%%%%%%%%%%%%%%%%%%%%%%%%%%%%%%%%%%%%%%%%%%%%%%%%%%%%%%%%%%%%%%%%%%%%%
% VAJE
%
% Sem vstavimo vaje s pomočjo okolja "vaja". Odgovor napišemo v vajo,
% v okolje "odgovor".

\begin{vaja}
V pravokotnem trikotniku s kotom $\beta= 90^{\circ}$, merita stranici $a=5$ in $b=13$. Z uporabo kotnih funkcij določi kota $\alpha$ in $\gamma$ na minuto natančno, ter izračunaj stranico $c$ na dve mesti natančno. Izračunaj še vrednosti $\sin\gamma$, $\cos\alpha$, $\tg\alpha$ in $\ctg\alpha$  na štiri decimalna mesta natančno.
	\begin{odgovor}
$\alpha=22^{\circ}37'$, $\gamma=67^{\circ}23'$, $c=12$, $\sin\gamma=\cos\alpha=0.9231$, $\tg\alpha=0.4167$ in $\ctg\alpha=2.4000$
	\end{odgovor}
\end{vaja}

\begin{vaja}
V enkokrakem trapezu merijo stranice $a=21$, $c=13$ in krak $b=7$. Izračunaj kot, ki ga oklepata stranica $a$ in $b$ na tri veljavna mesta natančno.
	\begin{odgovor}
$\alpha=55,2^{\circ}$
	\end{odgovor}
\end{vaja}

\begin{vaja}
V rombu merita diagonali $e=26,8$ in $f=12,5$. Izračunajte ostri kot romba na minuto natančno.
	\begin{odgovor}
$\alpha=50^{\circ}1'$
	\end{odgovor}
\end{vaja}

\begin{vaja}
  Narišite graf funkcije, ki je podan s predpisom $f(x) = 3sin(4x)$. Kje ima funkcija ničle, maksimume in minimume, kakšna sta definicijsko območje in zaloga vrednosti?

  \begin{odgovor}
Graf funkcije je takšen:\\
	\definecolor{qqwuqq}{rgb}{0.,0.6,0.}
	\definecolor{cqcqcq}{rgb}{0.75,0.75,0.75}
    \begin{tikzpicture}[line cap=round,line join=round,>=triangle 45,x=1.0cm,y=1.0cm]

	\draw [color=cqcqcq,, xstep=1.57cm,ystep=1.0cm] (-4.3,-4.3) grid (6.,4.3);
	\draw[->,color=black] (-4.3,0.) -- (6.,0.);
	\draw[shift={(-3.141592653589793,0)},color=black] (0pt,2pt) -- (0pt,-2pt) node[below] {\footnotesize $-\pi$};
	\draw[shift={(-1.5707963267948966,0)},color=black] (0pt,2pt) -- (0pt,-2pt) node[below] {\footnotesize $-\pi/2$};
	\draw[shift={(1.5707963267948966,0)},color=black] (0pt,2pt) -- (0pt,-2pt) node[below] {\footnotesize $\pi/2$};
	\draw[shift={(3.141592653589793,0)},color=black] (0pt,2pt) -- (0pt,-2pt) node[below] {\footnotesize $\pi$};
	\draw[shift={(4.71238898038469,0)},color=black] (0pt,2pt) -- (0pt,-2pt) node[below] {\footnotesize $3\pi/2$};
	
	\draw[->,color=black] (0.,-4.3) -- (0.,4.3);
	\foreach \y in {-4.,-3.,-2.,-1.,1.,2.,3.,4.}
	\draw[shift={(0,\y)},color=black] (2pt,0pt) -- (-2pt,0pt) node[left] {\footnotesize $\y$};
	\draw[color=black] (0pt,-10pt) node[right] {\footnotesize $0$};
	\clip(-4.3,-4.3) rectangle (6.,4.3);
	\draw[line width=1.5pt,color=qqwuqq,smooth,samples=100,domain=-4.3:6] plot(\x,{3.0*sin((4.0*(\x))*180/pi)});


\end{tikzpicture}
  \end{odgovor}
\end{vaja}

\begin{vaja}
  Podana je funkcija $f(x) = \frac{1}{2} cos(2x)$. Zapišite vsa presečišča grafa funkcije z osjo $x$ in z osjo $y$. Koordinate naj bodo izračunane točno. Zapišite tudi definicijsko območje in zalogo vrednosti. 

  \begin{odgovor}
    Rešitev bi bila tu.
  \end{odgovor}
\end{vaja}

%%%%%%%%%%%%%%%%%%%%%%%%%%%%%%%%%%%%%%%%%%%%%%%%%%%%%%%%%%%%%%%%%%%%%%
% Treba je zapredi datoteko z odgovori

\Closesolutionfile{odgovor}

%%%%%%%%%%%%%%%%%%%%%%%%%%%%%%%%%%%%%%%%%%%%%%%%%%%%%%%%%%%%%%%%%%%%%%
% Odgovori

\section{Odgovori}
\label{sec:sincos-odgovori}

% Vključimo odgovore.
\input{\datotekaOdgovori}


%%% Local Variables:
%%% mode: latex
%%% TeX-master: "vaje"
%%% End:
