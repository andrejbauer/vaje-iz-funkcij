\documentclass{book}

%%%%%%%%%%%%%%%%%%%%%%%%%%%%%%%%%%%%%%%%%%%%%%%%%%%%%%%%%%%%%%%%%%%%%%
% PAKETI

\usepackage[T1]{fontenc}
\usepackage[utf8]{inputenc}
\usepackage[slovene]{babel}

% Matematika
\usepackage{amsmath}
\usepackage{amssymb}
\usepackage{amsthm}

% Risanje slik
\usepackage{tikz}

% Vaje z rešitvami
\usepackage{answers}

% Naprednejši ukazi
\usepackage{xparse}

% Izbor pisave
\usepackage{mathpazo}
\usepackage[scaled=0.95]{helvet}
\usepackage{courier}
\linespread{1.05} % Pisava Palatino je boljša, če povečamo presledek med vrsticami.


%%%%%%%%%%%%%%%%%%%%%%%%%%%%%%%%%%%%%%%%%%%%%%%%%%%%%%%%%%%%%%%%%%%%%%
% NASLOV, AVTORJI
\title{Vaje iz funkcij}

\author{%
Andrej Bauer \and
Micka Kovačeva
}

%%%%%%%%%%%%%%%%%%%%%%%%%%%%%%%%%%%%%%%%%%%%%%%%%%%%%%%%%%%%%%%%%%%%%%
% OKOLJA ZA IZREKE, DEFINICIJE, ...


%%%%%%%%%%%%%%%%%%%%%%%%%%%%%%%%%%%%%%%%%%%%%%%%%%%%%%%%%%%%%%%%%%%%%%
% Okolje za vaje in rešitve

{\theoremstyle{definition}
\newtheorem{vaja}{Vaja}[chapter]
%\newtheorem{preodgovor}{Odgovor}
}

\Newassociation{odgovor}{Odg}{odgovor}
\renewcommand{\Odglabel}[1]{\textbf{Odgovor #1}}

%%%%%%%%%%%%%%%%%%%%%%%%%%%%%%%%%%%%%%%%%%%%%%%%%%%%%%%%%%%%%%%%%%%%%%
% MAKROJI

% !TeX root = vaje.tex
%%%%%%%%%%%%%%%%%%%%%%%%%%%%%%%%%%%%%%%%%%%%%%%%%%%%%%%%%%%%%%%%%%%%%%
% Množice

% Makro za množice je \set.
% Podamo mu lahko en izbirni argument v oglatih oklepajih []
% in enega ali dva obvezna argumenta v zavitih oklepajih {}.
% Izbirni argument je velikost zavitih oklepajev v zapisu množice.
% Dan je kot število od 0 do 4.
% Če ga ne podamo, se velikost zavitih oklepajev samodejno prilagodi vsebini.
% Če podamo samo en obvezni argument, se množica zapiše kot zaporedje elementov v zavitih oklepajih.
% Če podamo dva obvezna argumenta, se ta dva izpišeta, ločena z navpično črto in obdana z zavitimi oklepaji.
% Primer:
% \set{1, 2, 3}  izpiše  {1, 2, 3}.
% \set{x \in \RR}{x \geq 0}  izpiše  {x ∈ ℝ | x ≥ 0}.

\newcommand{\sizedescriptor}[2]
{
\ifthenelse{\equal{#1}{0}}{}{
\ifthenelse{\equal{#1}{1}}{\big}{
\ifthenelse{\equal{#1}{2}}{\Big}{
\ifthenelse{\equal{#1}{3}}{\bigg}{
\ifthenelse{\equal{#1}{4}}{\Bigg}{
#2}}}}}
}

\NewDocumentCommand{\set}
{O{auto} m G{\empty}}
{\sizedescriptor{#1}{\left}\{ {#2} \ifthenelse{\equal{#3}{}}{}{ \; \sizedescriptor{#1}{\middle}| \; {#3}} \sizedescriptor{#1}{\right}\}}


%%%%%%%%%%%%%%%%%%%%%%%%%%%%%%%%%%%%%%%%%%%%%%%%%%%%%%%%%%%%%%%%%%%%%%
% Številske množice

\newcommand{\NN}{\mathbb{N}}     % naravna števila
\newcommand{\NNz}{\mathbb{N}_0}  % naravna števila z ničlo
\newcommand{\ZZ}{\mathbb{Z}}     % cela števila
\newcommand{\QQ}{\mathbb{Q}}     % racionalna števila
\newcommand{\RR}{\mathbb{R}}     % realna števila


%%% Local Variables:
%%% mode: latex
%%% TeX-master: "vaje"
%%% End:


\begin{document}

\maketitle

%%%%%%%%%%%%%%%%%%%%%%%%%%%%%%%%%%%%%%%%%%%%%%%%%%%%%%%%%%%%%%%%%%%%%%
% KAZALO

\setcounter{tocdepth}{0} % Prikaži samo poglavja (nastavi na 1 za razdelke)

\tableofcontents

%%%%%%%%%%%%%%%%%%%%%%%%%%%%%%%%%%%%%%%%%%%%%%%%%%%%%%%%%%%%%%%%%%%%%%
% VSEBINA

\chapter{Uvod}
\label{cha:uvod}

Tu bo en lep uvod.

\begin{zgled}
  Reši enačbo $x^2 + 4 x + 1 = 0$.
\end{zgled}

\begin{resitev}
  Najprej dopolnimo do kvadrata,
  %
  \begin{equation*}
    (x + 2)^2 - 3 = 0,
  \end{equation*}
  %
  in razberemo rešitvi:
  %
  \begin{equation*}
    x_{1,2} = -2 + \pm \sqrt{3}.
  \end{equation*}
\end{resitev}

%%% Local Variables:
%%% mode: latex
%%% TeX-master: "vaje"
%%% End:

\chapter{Polinomi}
\label{cha:polinomi}

\section{Pregled snovi}
\label{sec:polinomi-pregled-snovi}


\subsection{Definicija polinoma}
Polinom je vsaka taka funkcija, ki jo lahko zapišemo v obliki:
\[
p(x)=a_nx^n + a_{n-1}x^{n-1}+ \cdots + a_2x^2 + a_1 x + a_0
\]
Pri tem naravno število n imenujemo stopnja polinoma oz. st(p), koeficienti $ a_j$ so realna števila, koeficient $ a_n$ (tj. tisti, pri najvišji potenci) imenujemo vodilni koeficient, $ a_0$ pa \textbf{prosti člen}.
\subsection{Posebni primeri}
\begin{itemize}
\item Polinom ničte stopnje je konstantni polinom, p(x)=a. V primeru a=0, ga imenujemo ničelni polinom.
\item Polinom prve stopnje je linearna funkcija, p(x)=kx + n
\item Polinom druge stopnje je kvadratna funkcija.
\end{itemize}
\subsection{Računske operacije s polinomi}
Vrednost polinoma v danem številu dobimo tako, da v polinom vstavimo to število. Tako kot vse funkcije lahko tudi polinome seštevamo, odštevamo, množimo in delimo (paziti moramo le na deljenje z 0). Veljajo pravila:

Recimo: $ p(x)=a_nx^n + a_{n-1}x^{n-1}+ \cdots + a_2x^2 + a_1 x + a_0$  in   q(x)=b_nx^n + b_{n-1}x^{n-1}+ \cdots + b_2x^2 + b_1 x + b_0$
$p(x) + q(x) = (p+q)(x) =(a_n + b_n)x^n + (a_{n-1}+b_{n-1)}x^{n-1}+ \cdots +(a_2 + b_2)x^2 +( a_1 + b_1) x + (a_0+b_0)$

Podobne formule dobimo tudi za $(p-q)(x), (p*q)(x), (\frac{p}{q})(x)$.
\subsection{Osnovni izrek o deljenju polinomov}
Vsak polinom p(x) (deljenec), lahko delimo s poljubnim neničelnim polinomom q(x) (deljitelj). Zapišemo ju lahko v obliko $p(x)=k(x)*q(x) + o(x)$, pri čemer je $k(x)$ polinom količnik, $o(x)$ pa polinom ostanek. Velja tudi $st(o)< st(q)$. Deljenje polinoma p(x) z polinomom q(x)=x-a, kjer je a neko število, lahko krajše napišemo s Hornerjevim algoritmom.

\subsection{Razcep polinoma, iskanje ničel}
V primeru, ko je število a ničla polinoma p, je ostanek pri deljenju polinoma p z (x-a) enak 0, zato lahko zapišemo $p(x)= k(x)*(x-a)$ postopek ponavljamo, sedaj na k(x). Tako lahko polinom zapišemo v \textbf{ničelno obliko}: 
\[
p(x)= a_n*(x-x_1)(x-x_2)(x-x_3)\cdots(x-x_n)
\]
Pri tem je $a_n$ vodilni koeficient, $x_i$ pa ničle polinoma p(x). V splošnem so te ničle lahko tudi nerealne. v tem primeru nastopajo v konjugiranih parih. Velja tudi st(p)=n. Ta zapis nam omogoča Gaussov izrek: Vsak nekonstanten polinom ima v $\mathbb{C}$ vsaj eno ničlo.

Za iskanje ničel poznamo več metod:
\begin{itemize}
\item Najlažji se zdi \textbf{razcep polinoma}, kjer se držimo pravil za razcep izrazov. Iz te oblike lahko enostavno razberemo ničle, a se metode ne da enostavno uporabiti pri vseh izrazih.
\item Ugibamo možne ničle in jih potem preverimo s \textbf{Hornerjevim algoritmom}. Pri tem upoštevamo dve pravili: cele ničle polinoma, ki ima le cele koeficiente, iščemo le med deljitelji prostega člena ter kandidati za racionalne ničle polinoma s celimi koeficienti so le ulomki, ki imajo v števcu deljitelj prostega člena, v imenovalcu pa deljitelj vodilnega člena.
\item Zgornji metodi nista vedno uporabni, ostaja nam možnost uporabe numeričnih metod (za približke ničel), npr. \textbf{metoda bisekcije}:\begin{enumerate}
\item Najprej poiščemo interval [a, b] na katerem polinom spremeni predznak (tj. da sta vrednosti polinoma v krajiščih različno predznačeni).
\item Izračunamo razpolovišče intervala: $c=\frac{a+b}{2}$
\item Izračunamo p(c). Če je vrednost enaka 0, smo dobili ničlo in postopek je uspešno zaključen. Če dobimo neničelno vrednost, moramo ugotoviti, na katerem od intervalov [a, c] ali [c, b] polinom spremeni predznak. Postopek ponovimo na tem intervalu. Večkrat kot ponovimo postopek, bolj natančen bo naš približek za ničlo. 
\end{enumerate}
\end{itemize}

\subsection{Graf polinoma}
Za risanje grafa moramo prej izračunati nekaj podatkov:
\begin{itemize}
\item Ničle polinoma, so presečišča grafa z abscisno osjo. Najdemo jih po zgoraj opisanih postopkih. Pri tem upoštevamo, da je pomembna tudi stopnja ničel. 
\begin{itemize}
\item V \textbf{enostavnih} ničlah (ničle prve stopnje) graf seka absciso pod nekim kotom.
\item V ničlah \textbf{sode} stopnje se graf abscise le dotakne, v tisti točki je tangent na graf enaka abscisi.
\item V ničlah \textbf{lihe} stopnje, večje od 1, graf seka absciso, a se ji v ničli lepo prilega in je vodoraven. Graf ima v tej ničli vodoravni prevoj.
\end{itemize}
\item \textbf{Začetna vrednost} je presečišče grafa z ordinatno osjo. Izračunamo jo kot p(0) in opazimo, da je enaka prostemu členu.
\item Zavedati se moramo, da je polinom zvezna funkcija, to pomeni, da je njen graf napretrgana krivulja. 
\item Ko gre $x\to \pm \infty$  je graf polinoma podoben grafu vodilnega člena, torej $a_nx^n$:
\begin{enumerate}
\item Če je n liho število, se predznak grafa zamenja v neskončnosti. Če je $a_ n$ pozitiven, graf narašča, če je negativen, pa pada.
\item Če je n sodo število, se predznak grafa ohrani. Če je $a_ n$ pozitiven, se graf prične in konča v pozitivni smeri, če pa je negativen, se prične in konča v $-\infty$.
\end{enumerate}
\item Za podrobnejši graf moramo pogledati tudi odvod polinoma p'(x). Stacionarne točke so vsi x, ki rešijo enačbo $p'(x)=0$. Če odvod v x spremeni predznak iz pozitivnega na negativni, ima graf v x lokalni maksimum, v obratnem primeru pa lokalni minimum. Če predznaka ne spremeni, garf v x nima ekstrema. Velja namreč, da na območju, kjer $p'(x) >0$ graf narašča, kjer $p'(x)<0$ pa graf pada.
\end{itemize}



\section{Vaje}
\label{sec:polinomi-funkcije-vaje}

%%%%%%%%%%%%%%%%%%%%%%%%%%%%%%%%%%%%%%%%%%%%%%%%%%%%%%%%%%%%%%%%%%%%%%
% Odpremo datoteko, v katero se bodo zapisali odgovori za
% to poglavje.

% Določimo ime datoteke, v katero se bodo pisali odgovori.
% Vsako poglavje mora imeti svojo datoteko.
\def\datotekaOdgovori{odgovori-polinomi}

% Odpremo datoteko z odgovori.
\Opensolutionfile{odgovor}[\datotekaOdgovori]

%%%%%%%%%%%%%%%%%%%%%%%%%%%%%%%%%%%%%%%%%%%%%%%%%%%%%%%%%%%%%%%%%%%%%%
% VAJE
%
% Sem vstavimo vaje s pomočjo okolja "vaja". Odgovor napišemo v vajo,
% v okolje "odgovor".

\begin{vaja}
  Poiščite ničle polinoma $x^3 + 3 x + 1$.

  \begin{odgovor}
    Grozna rešitev.
  \end{odgovor}
\end{vaja}

\begin{vaja}
  Še ena vaja.

  \begin{odgovor}
    Rešitev bi bila tu.
  \end{odgovor}
\end{vaja}

%%%%%%%%%%%%%%%%%%%%%%%%%%%%%%%%%%%%%%%%%%%%%%%%%%%%%%%%%%%%%%%%%%%%%%
% Treba je zapredi datoteko z odgovori

\Closesolutionfile{odgovor}

%%%%%%%%%%%%%%%%%%%%%%%%%%%%%%%%%%%%%%%%%%%%%%%%%%%%%%%%%%%%%%%%%%%%%%
% Odgovori

\section{Odgovori}
\label{sec:polinomi-odgovori}

% Vključimo odgovore.
\input{\datotekaOdgovori}


%%% Local Variables:
%%% mode: latex
%%% TeX-master: "vaje"
%%% End:

% !Tex root = vaje.tex
\chapter{Eksponentna in logaritemska funkcija}
\label{cha:exp-log}

\section{Pregled snovi}
\label{sec:exp-log-pregled-snovi}


\begin{center}
EKSPONENTNA FUNKCIJA
\end{center}

Eksponentna funkcija je preslikava $f: \mathbb{R} \longrightarrow \mathbb{R}^+$ oz. $f: x \longmapsto a^x \Leftrightarrow f(x) =~a^x$, pri čemer je $a > 0$ in $a \neq 1$, saj če bi bil $a=1$, bi dobili konstantno funkcijo	$f(x) = 1^x = 1$.
\\

\noindent Ločimo dva primera:
\begin{enumerate}
\item $a > 1$:
%
\begin{itemize}
\item Definicijsko območje so vsa realna števila: $\mathcal{D}_f = \mathbb{R}$.
\item Zaloga vrednosti so vsa pozitivna realna števila: $\mathcal{Z}_f = \mathbb{R}^+$.
\item Abscisna os je vodoravna asimptota, saj se približuje $y = 0$, vendar se je nikoli ne dotakne, ko gre $x \rightarrow -\infty$.
\item Začetna vrednost je pri vseh grafih enaka: $T(0,1)$, saj je $f(0) = a^0 = 1~\forall a$.
\item Funkcija je naraščajoča: $x_1 < x_2 \Rightarrow f(x_1) < f(x_2)$
\item Funkcija je injektivna: $x_1 \neq x_2 \Rightarrow f(x_1) \neq f(x_2)~\forall x_{1, 2} \in \mathbb{R}$.
%
\begin{figure}[h]
\centering
\definecolor{zzttqq}{rgb}{0.6,0.2,0.}
\definecolor{aqaqaq}{rgb}{0.6274509803921569,0.6274509803921569,0.6274509803921569}
\definecolor{qqwuqq}{rgb}{0.,0.39215686274509803,0.}
\definecolor{ccqqqq}{rgb}{0.8,0.,0.}
\begin{tikzpicture}[line cap=round,line join=round,>=triangle 45,x=1.0cm,y=1.0cm, scale=0.9]
\draw[->,color=black] (-3.4954873277047307,0.) -- (4.196120979865234,0.);
\foreach \x in {-3.,-2.,-1.,1.,2.,3.,4.}
\draw[shift={(\x,0)},color=black] (0pt,2pt) -- (0pt,-2pt);
\draw[->,color=black] (0.,-1.237832632360699) -- (0.,4.56310100742669);
\foreach \y in {-1.,1.,2.,3.,4.}
\draw[shift={(0,\y)},color=black] (2pt,0pt) -- (-2pt,0pt);
\clip(-3.4954873277047307,-1.237832632360699) rectangle (4.196120979865234,4.56310100742669);
\draw[line width=2.pt,color=ccqqqq,smooth,samples=100,domain=-3.4954873277047307:4.196120979865234] plot(\x,{2.0^((\x))});
\draw [line width=1.2pt,dash pattern=on 1pt off 1pt,color=aqaqaq,domain=-3.4954873277047307:4.196120979865234] plot(\x,{(-0.-0.*\x)/1.});
\draw [line width=1.2pt,dash pattern=on 1pt off 1pt,color=aqaqaq] (1.,0.)-- (1.,2.)-- (0.,2.);
\draw [line width=1.2pt,dash pattern=on 1pt off 1pt,color=aqaqaq] (-1.,0.)-- (-1.,0.5)-- (0.,0.5);
\begin{scriptsize}
\draw [fill=qqwuqq] (0.,1.) circle (2.0pt);
\draw[color=qqwuqq] (-0.498338280481239,1.1631093463290814) node {$T = (0, 1)$};
\draw[color=aqaqaq] (-2.4427252967803503,-0.30860900302438576) node {$asimptota$};
\draw [color=black] (1.,0.) circle (1.0pt);
\draw[color=black] (0.9733800688722319,-0.14747195747473607) node {1};
\draw [color=zzttqq] (1.,2.) circle (1.0pt);
\draw [color=black] (0.,2.) circle (1.0pt);
\draw[color=black] (-0.2082915984918688,2.0547343317038096) node {a};
\draw [color=black] (0.,0.) circle (1.0pt);
\draw[color=black] (-0.09012443175545873,-0.1367294877714261) node {0};
\draw [color=black] (-1.,0.) circle (1.0pt);
\draw[color=black] (-1.0462042353500494,-0.14747195747473607) node {-1};
\draw [color=zzttqq] (-1.,0.5) circle (1.0pt);
\draw [color=black] (0.,0.5) circle (1.0pt);
\draw[color=black] (0.2966044775637015,0.46484881561393276) node {1/a};
\end{scriptsize}
\end{tikzpicture}
\caption{Graf, ko $a>1$}
\end{figure}
\end{itemize}
%
\newpage
\item $0 < a < 1$:
\begin{itemize}
\item $\mathcal{D}_f = \mathbb{R}$
\item $\mathcal{Z}_f = \mathbb{R}^+$
\item Abscisna os je vodoravna asimptota, saj se približuje $y = 0$, vendar se je nikoli ne dotakne, ko gre $x \rightarrow \infty$.
\item Začetna vrednost: $T(0,1)$
\item Funkcija je padajoča: $x_1 < x_2 \Rightarrow f(x_1) > f(x_2)$
\item Funkcija je injektivna.
%
\begin{figure}[htb]
\centering
\definecolor{aqaqaq}{rgb}{0.6274509803921569,0.6274509803921569,0.6274509803921569}
\definecolor{zzttqq}{rgb}{0.6,0.2,0.}
\definecolor{qqzzqq}{rgb}{0.,0.6,0.}
\definecolor{ccqqqq}{rgb}{0.8,0.,0.}
\begin{tikzpicture}[line cap=round,line join=round,>=triangle 45,x=1.0cm,y=1.0cm, scale=0.8]
\draw[->,color=black] (-3.4398882747503903,0.) -- (4.784334733686732,0.);
\foreach \x in {-3.,-2.,-1.,1.,2.,3.,4.}
\draw[shift={(\x,0)},color=black] (0pt,2pt) -- (0pt,-2pt);
\draw[->,color=black] (0.,-1.7465646767819383) -- (0.,4.743220239093709);
\foreach \y in {-1.,1.,2.,3.,4.}
\draw[shift={(0,\y)},color=black] (2pt,0pt) -- (-2pt,0pt);
\clip(-3.4398882747503903,-1.7465646767819383) rectangle (4.784334733686732,4.743220239093709);
\draw[line width=2.pt,color=ccqqqq,smooth,samples=100,domain=-3.4398882747503903:4.784334733686732] plot(\x,{(1.0/2.0)^((\x))});
\draw [line width=1.2pt,dash pattern=on 1pt off 1pt,color=aqaqaq] (1.,0.)-- (1.,0.5)-- (0.,0.5);
\draw [line width=1.2pt,dash pattern=on 1pt off 1pt,color=aqaqaq] (-1.,0.)-- (-1.,2.)-- (0.,2.);
\draw [line width=1.2pt,dash pattern=on 1pt off 1pt,color=aqaqaq,domain=-3.4398882747503903:4.784334733686732] plot(\x,{(-0.-0.*\x)/1.});
\begin{scriptsize}
\draw [fill=qqzzqq] (0.,1.) circle (2.0pt);
\draw[color=qqzzqq] (0.4310099848073055,1.1422510866565136) node {T(0,1)};
\draw [color=black] (0.,0.) circle (1.0pt);
\draw[color=black] (-0.1203345744175295,-0.17867858648631718) node {0};
\draw [color=black] (1.,0.) circle (1.0pt);
\draw[color=black] (0.9593818540644391,-0.19016493147016789) node {1};
\draw [fill=zzttqq] (1.,0.5) circle (1.0pt);
\draw [color=black] (0.,0.5) circle (1.0pt);
\draw[color=black] (-0.2351980242560368,0.5679338374639785) node {a};
\draw [color=black] (-1.,0.) circle (1.0pt);
\draw[color=black] (-1.0392421731255876,-0.2016512764540186) node {-1};
\draw [fill=zzttqq] (-1.,2.) circle (1.0pt);
\draw [color=black] (0.,2.) circle (1.0pt);
\draw[color=black] (0.2931738450010968,2.015213305429167) node {1/a};
\draw[color=aqaqaq] (-2.1534176365591087,0.17739810801305458) node {asimptota};
\end{scriptsize}
\end{tikzpicture}
\caption{Graf, ko $0<a<1$}
\end{figure}
\end{itemize}
\end{enumerate}

\begin{center}
EKSPONENTNE ENAČBE
\end{center}

Eksponente enačbe so enačbe, ki imajo neznanke le v eksponentu. Najprej si osvežimo spomin, kako se računa s potencami:
$x, z \in \mathbb{R}, a, b > 0, a, b \neq 1$:
%
\begin{multicols}{2}
\begin{itemize}
\item $a^x \cdot a^z = a^{x + z}$
\item $a^x : a^z = a^{x - z}$
\item $(a^x)^z$
\item $a^x \cdot b^x = (a \cdot b)^x$
\item $\frac{a^x}{b^x} = (\frac{a}{b})^x$
\item $a^{-x} = \frac{1}{a^x}$
\item $a^0 = 1, a\neq 0$
\item$a^1 = a$
\item $a^{\frac{p}{q}} = \sqrt [q]a^p$
\item $(\sqrt[q]a^p)^n = \sqrt[q]{a^{np}}$
\end{itemize}
\end{multicols}

Eksponentne enačbe ločimo v pet skupin:
%
\begin{enumerate}
\item \textbf{osnovi potenc na obeh straneh enačbe sta enaki} oz. enačbo preuredimo tako, da imamo na obeh straneh enaki osnovi: $a^{f(x)} = a^{g(x)}$. Enačaj bo veljal le, če bosta eksponenta enaka: $f(x) = g(x)$.
\item \textbf{potenci imata različni osnovi, vendar enak eksponent:} $a^{f(x)} =b^{f(x)}$. Enačaj bo v tem primeru veljal le, če bosta eksponenta enaka nič, saj grafa eksponentne funkcije z različnima osnovama potekata skozi eno skupno točko in sicer $T(0, 1) \Rightarrow f(x) = 0$ (glej \ref{fig: grafa}) ali:
%
\begin{equation*}
\bigg(\frac{a}{b}\bigg)^{f(x)}=1 \Rightarrow \bigg(\frac{a}{b}\bigg)^{f(x)} = \bigg(\frac{a}{b}\bigg)^0 \text{po zgornji točki}\Rightarrow f(x) = 0.
\end{equation*}
%
\begin{figure}[h]
\centering
\definecolor{ccqqqq}{rgb}{0.8,0.,0.}
\definecolor{qqwuqq}{rgb}{0.,0.39215686274509803,0.}
\begin{tikzpicture}[line cap=round,line join=round,>=triangle 45,x=1.0cm,y=1.0cm, scale=0.8]
\draw[->,color=black] (-4.394698325375187,0.) -- (4.496895020831907,0.);
\foreach \x in {-4.,-3.,-2.,-1.,1.,2.,3.,4.}
\draw[shift={(\x,0)},color=black] (0pt,2pt) -- (0pt,-2pt) node[below] {\footnotesize $\x$};
\draw[->,color=black] (0.,-1.4455497947855014) -- (0.,5.570861155782946);
\foreach \y in {-1.,1.,2.,3.,4.,5.}
\draw[shift={(0,\y)},color=black] (2pt,0pt) -- (-2pt,0pt) node[left] {\footnotesize $\y$};
\draw[color=black] (0pt,-10pt) node[right] {\footnotesize $0$};
\clip(-4.394698325375187,-1.4455497947855014) rectangle (4.496895020831907,5.570861155782946);
\draw[line width=2.pt,color=qqwuqq,smooth,samples=100,domain=-4.394698325375187:4.496895020831907] plot(\x,{2.0^((\x))});
\draw[line width=2.pt,color=ccqqqq,smooth,samples=100,domain=-4.394698325375187:4.496895020831907] plot(\x,{(1.0/2.0)^((\x))});
\begin{scriptsize}
\draw [fill=black] (0.,1.) circle (2.0pt);
\draw[color=black] (0.6223459649427827,1.0567631371428918) node {T(0,1)};
\end{scriptsize}
\end{tikzpicture}
\caption{Grafa dveh eksponentnih funkcij}
\label{fig: grafa}
\end{figure}

\item \textbf{neznanka v eksponentu potence je pomnožena z različnimi koeficienti:} $a^{bx + c}$. Enačbo poenostaviš tako, da vpelješ novo spremenljivko $u$, kjer bo eksponent najmanjši faktor ter računaš naprej kvadratno enačbo z $u$-ji.
\item \textbf{enačbe, kjer nastopajo eksponentne in neeksponentne funkcije:} $a^{f(x)} = b$, kjer $b$ ni potenca števila $a$ ali $a^{f(x)} = x + c$. V takem primeru najprej narišemo grafa obeh funkcij, ki ju vsebuje enačba. Z grafa prebereš rešitve. 
\item \textbf{enačbe, kjer nastopajo vsote ali razlike potenc z enakimi osnovami} rešujemo  z izpostavljanjem skupnega faktorja. Tako enačbo prevedemo na že znane eksponntne enačbe.
\end{enumerate}

\begin{center}
EKSPONENTNE NEENAČBE
\end{center}
Eksponentne neenačbe so podobne eksponetnim enačbam, le da imajo vmes neenačaj. Rešitve neenačb so intervali, unije intervalov ali prazna množica. Prav tako kot pri eksponentnih enačbah lahko rešujemo neenačbe na različne načine:
%
\begin{enumerate}
\item \textbf{osnovi potenc na obeh straneh neenačbe sta enaki in $a>1$} oz. neenačbo preuredimo tako, da imamo na obeh straneh enaki osnovi: $a^{x_1} > a^{x_2} \Rightarrow x_1 > x_2$.
\item \textbf{potenci imata različni osnovi, vendar enak eksponent in $a>1$:} $a^x < b^x$. Delimo neenačbo s tisto potenco, ki ima manjšo osnovo.
\item \textbf{osnovi potenc na obeh straneh enačbe sta enaki in $0<a<1$:}
\\
$a^{x_1}<a^{x_2} \Rightarrow x_1 > x_2$.
\end{enumerate}

\begin{center}
LOGARITEMSKA FUNKCIJA
\end{center}

Ker je eksponentna funkcija  $f: x \mapsto a^x$ bijektivna, je obrnljiva, kar pa pomeni, da ima inverz $f^{-1}$. Njena inverzna funkcija je logaritemska funkcija z enako osnovo. Logaritemska funkcija z osnovo $a$, kjer je $a > 0$ in $a \neq 1$ je preslikava $f: x \mapsto \log_a{x}$, za vse $x > 0$. Torej velja:
\begin{equation*}
a^x=y \Leftrightarrow x=\log_a{y}
\end{equation*}
%
Iz te definicije sledi tudi: 
\begin{equation*}
a^{\log_a{y}} = y \qquad \text{in} \qquad \log_a{a^x} = x
\end{equation*}
%
Gaf logaritemske funkcije dobimo tako, da čez simetralo lihih kvadrantov $y=x$ prezrcalimo eksponentno funkcijo, saj je njen inverz. Tako kot pri eksponentni funkciji tudi tu ločimo dva preimera:
\begin{enumerate}
\item $a>1$:
%
\begin{itemize}
\item  $\mathcal{D}_f = \mathbb{R}^+$
\item $\mathcal{Z}_f = \mathbb{R}$
\item Ordinatna os je navpična asimptota.
\item Vsi grafi gredo skozi točko $T(1, 0)$, saj je ena ničla logaritma: $\log_a{1}=\log_a{a^0}
=0$.
\item Funkcija je naraščajoča: $x_1 < x_2 \Rightarrow f(x_1) < f(x_2)$
\item Funkcija je neomejena: ko vrednosti spremenljivke $x$ naraščajo od $-\infty$ do $+\infty$, graf funkcije narašča od minus neskončno do neskončnosti, zato funkcija navzdol in navzgor ni omejena.
\item Funkcija je pozitivna na intervalu $(1, \infty)$ in negativna na $(0, 1)$.
%
\begin{figure}[h]
\centering
\definecolor{uuuuuu}{rgb}{0.26666666666666666,0.26666666666666666,0.26666666666666666}
\definecolor{qqwuqq}{rgb}{0.,0.39215686274509803,0.}
\definecolor{ccqqqq}{rgb}{0.8,0.,0.}
\definecolor{aqaqaq}{rgb}{0.6274509803921569,0.6274509803921569,0.6274509803921569}
\begin{tikzpicture}[line cap=round,line join=round,>=triangle 45,x=1.0cm,y=1.0cm, scale=0.7]
\draw[->,color=black] (-5.907272727272731,0.) -- (8.412727272727277,0.);
\foreach \x in {-5.,-4.,-3.,-2.,-1.,1.,2.,3.,4.,5.,6.,7.,8.}
\draw[shift={(\x,0)},color=black] (0pt,2pt) -- (0pt,-2pt);
\draw[->,color=black] (0.,-5.101818181818178) -- (0.,5.698181818181817);
\foreach \y in {-5.,-4.,-3.,-2.,-1.,1.,2.,3.,4.,5.}
\draw[shift={(0,\y)},color=black] (2pt,0pt) -- (-2pt,0pt);
\clip(-5.907272727272731,-5.101818181818178) rectangle (8.412727272727277,5.698181818181817);
\draw[line width=2.pt,dash pattern=on 5pt off 5pt,color=aqaqaq,smooth,samples=100,domain=-5.907272727272731:8.412727272727277] plot(\x,{2.0^((\x))});
\draw [line width=2.pt,dash pattern=on 5pt off 5pt,color=aqaqaq,domain=-5.907272727272731:8.412727272727277] plot(\x,{(-0.--1.*\x)/1.});
\draw[line width=2.pt,color=ccqqqq,smooth,samples=4001,domain=0.01:100] plot(\x,{log2(\x)});
\draw [line width=1.2pt,dash pattern=on 2pt off 2pt,color=aqaqaq] (0.,-5.101818181818178) -- (0.,5.698181818181817);
\draw [line width=1.2pt,dash pattern=on 2pt off 2pt,color=aqaqaq] (0.,-1.)-- (0.4927272727272728,-1.0211387670581502)-- (0.4927272727272728,0.);
\draw [line width=1.2pt,dash pattern=on 2pt off 2pt,color=aqaqaq] (0.,1.)-- (2.,1.)-- (2.,0.);
\begin{scriptsize}
\draw[color=aqaqaq] (-3.7072727272727297,0.5881818181818196) node {eksponentna};
\draw [color=black] (0.,0.) circle (1.0pt);
\draw[color=black] (-0.16727272727272757,-0.25181818181818) node {0};
\draw[color=aqaqaq] (-3.107272727272729,-3.9118181818181785) node {y = x};
\draw [fill=qqwuqq] (1.000000000020259,2.922742273521049E-11) circle (2.0pt);
\draw[color=qqwuqq] (1.2527272727272731,-0.31181818181818) node {T(1, 0)};
\draw[color=aqaqaq] (0.6327272727272728,5.408181818181817) node[below] {asimptota};
\draw [fill=uuuuuu] (0.,-1.) circle (1.0pt);
\draw[color=uuuuuu] (-0.3672727272727277,-0.9318181818181797) node {-1};
\draw [fill=uuuuuu] (0.4927272727272728,0.) circle (1.0pt);
\draw[color=uuuuuu] (0.4127272727272727,0.3481818181818197) node {1/a};
\draw [fill=uuuuuu] (0.,1.) circle (1.0pt);
\draw[color=uuuuuu] (-0.4072727272727277,1.1281818181818193) node {1};
\draw [fill=uuuuuu] (2.,0.) circle (1.5pt);
\draw[color=uuuuuu] (1.9727272727272736,-0.33181818181818) node {a};
\end{scriptsize}
\end{tikzpicture}
\caption{Graf logaritma $a>1$}
\end{figure}
\end{itemize}

\item $0 < a < 1$:
%
\begin{itemize}
\item  $\mathcal{D}_f = \mathbb{R}^+$
\item $\mathcal{Z}_f = \mathbb{R}$
\item Ordinatna os je navpična asimptota.
\item Vsi grafi gredo skozi točko $T(1, 0)$, saj je ena ničla logaritma: $\log_a{1}=0$.
\item Funkcija je padajoča: $x_1 < x_2 \Rightarrow f(x_1) > f(x_2)$
\item Funkcija je neomejena.
\item Funkcija je negativna na intervalu $(1, \infty)$ in pozitivna na $(0, 1)$.
%
\begin{figure}[h!]
\centering
\definecolor{qqwuqq}{rgb}{0.,0.39215686274509803,0.}
\definecolor{ccqqqq}{rgb}{0.8,0.,0.}
\definecolor{aqaqaq}{rgb}{0.6274509803921569,0.6274509803921569,0.6274509803921569}
\begin{tikzpicture}[line cap=round,line join=round,>=triangle 45,x=1.0cm,y=1.0cm]
\draw[->,color=black] (-5.006149047230808,0.) -- (7.034935859421989,0.);
\foreach \x in {-5.,-4.,-3.,-2.,-1.,1.,2.,3.,4.,5.,6.,7.}
\draw[shift={(\x,0)},color=black] (0pt,2pt) -- (0pt,-2pt);
\draw[->,color=black] (0.,-4.164698738644652) -- (0.,5.336995356688894);
\foreach \y in {-4.,-3.,-2.,-1.,1.,2.,3.,4.,5.}
\draw[shift={(0,\y)},color=black] (2pt,0pt) -- (-2pt,0pt);
\clip(-5.006149047230808,-4.164698738644652) rectangle (7.034935859421989,5.336995356688894);
\draw[line width=2.pt,dash pattern=on 4pt off 4pt,color=aqaqaq,smooth,samples=100,domain=-5.006149047230808:7.034935859421989] plot(\x,{(1.0/2.0)^((\x))});
\draw [line width=1.2pt,dash pattern=on 2pt off 2pt,color=aqaqaq] (0.,-4.164698738644652) -- (0.,5.336995356688894);
\draw [line width=2.pt,dash pattern=on 4pt off 4pt,color=aqaqaq,domain=-5.006149047230808:7.034935859421989] plot(\x,{(-0.--1.*\x)/1.});
\draw[line width=2.pt,color=qqwuqq,smooth,samples=4001,domain=0.01:100] plot(\x,{ln((\x))/ln(0.5)});
\draw [line width=1.2pt,dash pattern=on 2pt off 2pt,color=aqaqaq] (0.,1.)-- (0.4996206865294056,1.0010948826822181)-- (0.5098786753699146,0.);
\draw [line width=1.2pt,dash pattern=on 2pt off 2pt,color=aqaqaq] (0.,-1.)-- (2.000001852694624,-1.0000013364360543)-- (2.,0.);
\begin{scriptsize}
\draw[color=aqaqaq] (-1.3231915129333742,4.2859230010104055) node {eksponentna};
\draw [color=black] (0.,0.) circle (1.0pt);
\draw[color=black] (-0.16280763226431988,-0.25470957552067003) node {0};
\draw[color=aqaqaq] (0.6275987792059056,5.093146570171485) node {asimptota};
\draw[color=aqaqaq] (-2.6517469705109873,-3.096809225275306) node {y=x};
\draw [fill=qqwuqq] (1.,0.) circle (2.0pt);
\draw[color=qqwuqq] (1.4852738214395544,0.6029654667129775) node {$T = (1, 0)$};
\draw [fill=black] (0.,1.) circle (1.0pt);
\draw[color=black] (-0.31416205148202264,1.0065772512935174) node {1};
\draw [fill=black] (0.5098786753699146,0.) circle (1.0pt);
\draw[color=black] (0.47624435998820286,-0.20425810244810255) node {a};
\draw [fill=black] (0.,-1.) circle (1.0pt);
\draw[color=black] (-0.2468934207185992,-0.9442130408457592) node {-1};
\draw [fill=aqaqaq] (2.,0.) circle (1.0pt);
\draw[color=black] (2.157960129073789,-0.2715267332115259) node[right] {1/a};
\end{scriptsize}
\end{tikzpicture}
\end{figure}
\end{itemize}
\end{enumerate}

\newpage
Za računanje z logaritmi velja nekaj pravil:
\begin{multicols}{2}
\begin{itemize}
\item $y=a^x \Rightarrow x=\log_a{y}$
\item $\log_a{a^x} = x$
\item $\log_a{a}=1$
\item $\log_a{1}=0$
\item$a^{\log_a{y}}=y$
\item $\log_a{(x\cdot z)}=\log_a{x} + \log_a{z}$
\item $\log_a{x^r}=r \cdot \log_a{x}; \qquad r \in \mathbb{R}$
\item $\log_a{(\frac{x}{z})}=\log_a{x}-\log_a{z}$
\item $\log_a{x}=\frac{\log_b{x}}{\log_b{a}}$
\end{itemize}
\end{multicols}

V matematiki največ uporabljamo logaritma z osnovo 10, ki mu pravimo desetiški logaritem in logaritem z osnovo $e$, ki ga imenujemo naravni logaritem. Po dogovoru pišemo:
%
\begin{equation*}
\log_{10}{x} = \log{x} \qquad \text{in} \qquad \log_e{x} = \ln x
\end{equation*}

Logaritemske enačbe in neenačbe rešujemo s pravili za logaritmiranje.


\section{Vaje}
\label{sec:exp-log-vaje}

%%%%%%%%%%%%%%%%%%%%%%%%%%%%%%%%%%%%%%%%%%%%%%%%%%%%%%%%%%%%%%%%%%%%%%
% Odpremo datoteko, v katero se bodo zapisali odgovori za
% to poglavje.

% Določimo ime datoteke, v katero se bodo pisali odgovori.
% Vsako poglavje mora imeti svojo datoteko.
\def\datotekaOdgovori{odgovori-explog}

% Odpremo datoteko z odgovori.
\Opensolutionfile{odgovor}[\datotekaOdgovori]

%%%%%%%%%%%%%%%%%%%%%%%%%%%%%%%%%%%%%%%%%%%%%%%%%%%%%%%%%%%%%%%%%%%%%%
% VAJE
%
% Sem vstavimo vaje s pomočjo okolja "vaja". Odgovor napišemo v vajo,
% v okolje "odgovor".

\begin{vaja}
 Poenostavi naslednji izraz: \(\sqrt{x^{\frac{10}{3}}y^{\frac{4}{2}}}(x^{2}y^{\frac{1}{2}})^{4}\).

  \begin{odgovor}
   Vsa števila moramo najprej napisati v taki obliki, da so potence realna števila. 
	$$\sqrt{x^{\frac{10}{3}}y^{\frac{4}{2}}}(x^{2}y^{\frac{1}{2}})^{4}=(x^{\frac{10}{3}}y^{\frac{4}{2}})^{\frac{1}{2}}(x^{2}y^{\frac{1}{2}})^{4}=x^{\frac{10}{6}}yx^{8}y^{2}=y^{3}x^{\frac{29}{3}}$$
  \end{odgovor}
\end{vaja}

\begin{vaja}
  Čim bolj poenostavi naslednji izraz: \(\,x^{\frac{5}{3}}y^{\frac{8}{7}}\sqrt{\sqrt{x^{\frac{1}{7}}y^{\frac{14}{3}}}}\).

  \begin{odgovor}
   Rešitev: $x^{\frac{1}{28}} y^{\frac{97}{42}}$.
  \end{odgovor}
\end{vaja}

\begin{vaja}
  Napiši vse rešitve naslednje enačbe \( 8^m=32\).

  \begin{odgovor}
   Enačbo želimo najprej  preurediti v obliko $a^{f(x)}=b^{g(x)}$, potem pa imamo 2 različni možnosti:
	\begin{itemize}
		\item[a)] $a=b$ 

			v tem primeru velja: $a^{f(x)}=a^{g(x)}$, kar pa drži natanko tedaj, ko je $f(x)=g(x)$ saj je $a^x$ injektivna.
			
		\item[b)] $a\neq b$:
			
			v tem primeru pa velja $a^{f(x)}=b^{g(x)}$. Ker sta a in b različna, se te enačbe lotimo s pomočjo logartima.
			$$a^{f(x)}=b^{g(x)}\implies f(x)\log(a)=g(x)\log(b)$$
	\end{itemize}
	
	Ta primer je tipa a), saj :
	\begin{align*}
		8^m&=32\\
		(2^3)^m&=2^5\\
		2^{3m}&=2^5\\
		3m&=5\\
		m&=\frac{5}{3}
	\end{align*}
  \end{odgovor}
\end{vaja}


\begin{vaja}
	Določi vse rešitve za katere velja: \(3^{x^2-4x}=1/81\).
  \begin{odgovor}$x_{1,2}=2$.
  \end{odgovor}
\end{vaja}


\begin{vaja}
  Reši enačbo $4^{x-1}=3^x$.

  \begin{odgovor}
	\begin{align*}
	 4^{x-1}&=3^x\\
	 (2^2)^{x-1}&=3^x\\
	 2^{2x-2}&=3^x\\
	 (2x-2)\log(2)&=x\log(3)\\
	 2x\log2-x\log(3)&=2\log2\\
	 x(2\log2-\log3)&=2\log2\\
	 x&=\frac{2\log2}{2\log2-\log3}\\
	 x&=\frac{2\log2}{\log{\frac{4}{3}}}
	\end{align*}    
  \end{odgovor}
\end{vaja}

\begin{vaja}
	Izračunaj rešitve enačbe \( 15^{x-2}=5^x\).
  \begin{odgovor}
   	$x=2+\frac{\log25}{\log3}$
  \end{odgovor}
\end{vaja}

\begin{vaja}
	Kateri x-i zadoščajo enačbi \(-7\cdot 3^x+9^x=-12\).
  \begin{odgovor}
   	Enačbo najprej prepišemo v tako obliko, da bodo v osnovi eksponenta praštevila in poskusimo:
	\begin{align*}
	-7 \cdot 3^x+9^x&=-12\\
	-7\cdot 3^x+3^{2x}&=-12\\
	-7\cdot 3^x+(3^x)^2&=-12\quad \text{vpeljimo novo neznanko } 3^x=y\\
	-7y+y^2&=-12\\
	y^2-7y+12&=0\\
	y_{1}=4 \lor y_{2}=3 & \quad \text{pretvorimo sedaj nazaj na prvotno spremenljivko}\\
	x_{1}&=\log_{3}{4}\\
	x_{2}&=\log_{3}{3}=1
	\end{align*}
  \end{odgovor}
\end{vaja}

\begin{vaja}
	Napiši rešitve enačbe $-16 \cdot 2^x+4^x=-64$.
  \begin{odgovor}
   	x=3.
  \end{odgovor}
\end{vaja}

\begin{vaja}
Določi interval, kjer bo $e^x < 16$.
  \begin{odgovor}
   Najprej enačbo logaritmiramo z $\ln$ in dobimo:
\begin{align*}
\ln{e^x}&< \ln 16 \\
x &< \ln 16
\end{align*}
  \end{odgovor}
\end{vaja}


\begin{vaja}
  \begin{odgovor}
   
  \end{odgovor}
\end{vaja}
%%%%%%%%%%%%%%%%%%%%%%%%%%%%%%%%%%%%%%%%%%%%%%%%%%%%%%%%%%%%%%%%%%%%%%
% Treba je zapredi datoteko z odgovori

\Closesolutionfile{odgovor}

%%%%%%%%%%%%%%%%%%%%%%%%%%%%%%%%%%%%%%%%%%%%%%%%%%%%%%%%%%%%%%%%%%%%%%
% Odgovori

\section{Odgovori}
\label{sec:explog-odgovori}

% Vključimo odgovore.
\input{\datotekaOdgovori}


%%% Local Variables:
%%% mode: latex
%%% TeX-master: "vaje"
%%% End:

% !TeX root = vaje.tex

\chapter{Kotne funkcije}
\label{cha:sin-cos}

\section{Pregled snovi}
\label{sec:sin-cos-pregled-snovi}

\subsection{Definicije kotnih funkcij v pravokotnem trikotniku}

Funkcije sinus, kosinus, tangens in kotagens imenujemo kotne oziroma trigonometrične funkcije. V nadaljevanju jih bomo označevali z $\sin x$, $\cos x$, $\tg x$ in $\ctg x$. V pravokotnem trikotniku jih definiramo z razmerji stranic in so odvisne od velikosti kota. 

\begin{equation*}
\begin{split}
\sin \alpha &= \frac{\text{nasprotna kateta}}{\text{hipotenuza}} \\
\cos \alpha &= \frac{\text{priležna kateta}}{\text{hipotenuza}}  
\end{split}
\quad
\begin{split}
\tg \alpha &= \frac{\text{nasprotna kateta}}{\text{priležna kateta}}\\
\ctg \alpha &= \frac{\text{priležna kateta}}{\text{nasprotna kateta}}
\end{split}
\end{equation*}

\definecolor{qqwuqq}{rgb}{0.,0.39215686274509803,0.}
\definecolor{zzttqq}{rgb}{0.6,0.2,0.}
\definecolor{xdxdff}{rgb}{0.49019607843137253,0.49019607843137253,1.}
\definecolor{ududff}{rgb}{0.30196078431372547,0.30196078431372547,1.}
\begin{tikzpicture}[line cap=round, line join=round,>=triangle 45,x=1.0cm,y=1.0cm]
\clip(-4.3,-2.46) rectangle (7.06,6.3);
\fill[line width=2.pt,color=zzttqq,fill=zzttqq,fill opacity=0.10000000149011612] (-2.32,0.9) -- (4.040307722752683,1.115323801513878) -- (1.42,4.14) -- cycle;
\draw[line width=2.pt,color=qqwuqq,fill=qqwuqq,fill opacity=0.10000000149011612] (1.152776418235458,3.908501496011466) -- (1.3842749222239916,3.641277914246924) -- (1.6514985039885335,3.872776418235458) -- (1.42,4.14) -- cycle; 
\draw [shift={(-2.32,0.9)},line width=2.pt,color=qqwuqq,fill=qqwuqq,fill opacity=0.10000000149011612] (0,0) -- (1.9389682973337594:1.2) arc (1.9389682973337594:40.902716394791724:1.2) -- cycle;
\draw [line width=2.pt,color=zzttqq] (-2.32,0.9)-- (4.040307722752683,1.115323801513878);
\draw [line width=2.pt,color=zzttqq] (4.040307722752683,1.115323801513878)-- (1.42,4.14);
\draw [line width=2.pt,color=zzttqq] (1.42,4.14)-- (-2.32,0.9);
\draw [fill=ududff] (-2.32,0.9) circle (2.5pt);
\draw[color=ududff] (-2.44,1.25) node {$A$};
\draw [fill=ududff] (1.42,4.14) circle (2.5pt);
\draw[color=ududff] (1.56,4.51) node {$C$};
\draw [fill=xdxdff] (4.040307722752683,1.115323801513878) circle (2.5pt);
\draw[color=ududff] (4.18,1.49) node {$B$};
\draw[color=zzttqq] (1.1,0.49) node {\text{hipotenuza}};
\draw[color=zzttqq] (4.14,3.16) node {\text{nasprotna kateta}};
\draw[color=zzttqq] (-1.36,3.06) node {\text{priležna kateta}};
\draw[color=qqwuqq] (-0.68,1.47) node {$\alpha$};
\end{tikzpicture}

\begin{zgled}
V pravokotnem trikotniku, kjer je kot $\gamma= 90^{\circ}$, merijo stranice $a=9$, $b=12$ in $c=15$. Izračunaj vrednosti $\cos\alpha$, $\sin\alpha$, $\tg\alpha$, $\tg\beta$ in izračunaj kota $\alpha$ in $\beta$. Določi še vrednost $\cos\beta$. Kaj opaziš? Rezultate razmerij zaokrožuj na štiri decimalna mesta natančno, kotov pa na dve decimalni mesti natančno.
\end{zgled}

\begin{resitev}
Glede na kot $\alpha$ je $b$ priležna kateta, $a$ nasprotna kateta in $c$ hipotenuza. Po definiciji:
\begin{equation*}
\cos\alpha= \frac{b}{c}= 0,8000 \quad \quad
\sin\alpha = \frac{a}{c}= 0,6000 \quad \quad
\tg \alpha= \frac{a}{b}= 0,7500
\end{equation*}
Glede na kot $\beta$ je $a$ priležna kateta, $b$ nasprotna kateta in $c$ hipotenuza. Po definiciji:
\begin{equation*}
\tg\beta= \frac{b}{a}= 1,3333 \quad \quad
\cos\beta = \frac{a}{c}= 0,6000
\end{equation*}
Opazimo, da imata $\cos\beta$ in $\sin\alpha$ isto vrednost, saj sta kota komplementarna.
Izračunajmo še kota. Ugotovili smo že, da velja $\cos\alpha= \frac{b}{c}$. Iz tega sledi:
\begin{equation*}
\alpha= \arccos \left(\frac{b}{c}\right) = 36,87^{\circ}
\end{equation*}
Enako še za $\beta$:
\begin{equation*}
\beta= \arctg \left(\frac{b}{a}\right) = 53,13^{\circ}
\end{equation*}
Pomoč: za izračun $\arccos x$ v kalkuator vtipkajte $\cos^{-1}x$.
\end{resitev}

\subsection{Zveze med kotnimi funkcijami}

Naslednje enačbe uporabljamo pri poenostavljanju in reševanju trigonometričnih enačb. Funkciji tangens in kotangens lahko definiramo s sinusom in kosinusom:
\begin{equation*}
\tg \alpha = \frac{\sin \alpha}{\cos \alpha}, \quad
\ctg \alpha = \frac{\cos \alpha}{\sin \alpha} \quad 
\Rightarrow \ctg \alpha \cdot \tg \alpha = 1
\end{equation*}
Funkciji sinus in kosinus pa povezuje enačba
\begin{equation*}
\sin^2 \alpha + \cos^2 \alpha = 1.
\end{equation*}
Če enačbo delimo s $\cos^2\alpha$ in s $\sin^2\alpha$  pa dobimo povezavi
\begin{equation*}
1+ \tg^2 \alpha = \frac{1}{\cos^2\alpha}\quad \text{in} \quad
1+ \ctg^2 \alpha= \frac{1}{\sin^2\alpha}.
\end{equation*}

V pravokotnem trikotniku, kjer je $\gamma={90}^{\circ}$, sta kota $\alpha$ in $\beta$ komplementarna. Zato zanju velja $\alpha+\beta= \frac{\pi}{2}$. Tako dobimo zveze:
\begin{align*}
\sin\alpha &= \cos(\frac{\pi}{2}- \alpha) \\
\cos\alpha &= \sin(\frac{\pi}{2}- \alpha) \\
\tg \alpha &= \ctg(\frac{\pi}{2}- \alpha) \\
\ctg\alpha &= \tg(\frac{\pi}{2}- \alpha).
\end{align*}

\subsection{Definicije kotnih funkcij na enotski krožnici}

Kotne funkcije definiramo tudi s pomočjo enotske krožnice. Poltrak iz koordinatnega izhodišča s pozitivnim delom abscisne osi določa kot $\alpha$. S pravokotnico na $x$-os skozi presečišče krožnice in poltraka dobimo pravokotni trikotnik s hipotenuzo dolžine 1. Abscisa presečišča $T$ tako predstavlja vrednost funkcije $\cos\alpha$, ordinata pa vrednost funkcije $\sin\alpha$. Na sliki rdeči znaki minus in plus ponazarjajo, v katerih kvadrantih je funkcija kosinus negativna in v katerih pozitivna. Enako označuje modra barva predznak funkcije sinus.

\

\definecolor{qqwuqq}{rgb}{0.,0.39215686274509803,0.}
\definecolor{ffqqqq}{rgb}{1.,0.,0.}
\definecolor{qqqqff}{rgb}{0.,0.,1.}
\definecolor{wqwqwq}{rgb}{0.3764705882352941,0.3764705882352941,0.3764705882352941}
\definecolor{uuuuuu}{rgb}{0.26666666666666666,0.26666666666666666,0.26666666666666666}
\begin{tikzpicture}[line cap=round,line join=round,>=triangle 45,x=3.0cm,y=3.0cm]
\draw[->,color=black] (-1.81936,0.) -- (2.0581866666666713,0.);
\foreach \x in {-1.5,-1.,-0.5,0.5,1.,1.5,2.}
\draw[shift={(\x,0)},color=black] (0pt,-2pt);
\draw[color=black] (1.94896,0.027306666666666635) node [anchor=south west] { x};
\draw[->,color=black] (0.,-1.4393955555555527) -- (0.,1.5506844444444439);
\foreach \y in {-1.,-0.5,0.5,1.,1.5}
\draw[shift={(0,\y)},color=black] (-2pt,0pt);
\draw[color=black] (0.034133333333333356,1.4004977777777774) node [anchor=west] { y};
\clip(-1.81936,-1.4393955555555527) rectangle (2.0581866666666713,1.5506844444444439);
\draw [shift={(0.,0.)},line width=2.pt,color=qqwuqq,fill=qqwuqq,fill opacity=0.10000000149011612] (0,0) -- (0.:0.2048) arc (0.:56.44300446165065:0.2048) -- cycle;
\draw [line width=2.pt,color=wqwqwq] (0.,0.) circle (3.0cm);
\draw [line width=2.pt,domain=0.0:2.0581866666666713] plot(\x,{(-0.--0.8333363648462626*\x)/0.5527662281876642});
\draw [line width=2.pt,color=qqqqff] (0.5527662281876642,0.8333363648462626)-- (0.5527662281876642,0.);
\draw [line width=2.pt,color=ffqqqq] (0.5527662281876642,0.)-- (0.,0.);
\draw [color=qqqqff](0.19576,1.4960711111111107) node[anchor=north west] {$\mathbf{+}$};
\draw [color=ffqqqq](1.5120533333333375,0.24476444444444524) node[anchor=north west] {$\mathbf{+}$};
\draw [color=qqqqff](-0.30384,1.4960711111111107) node[anchor=north west] {$\mathbf{+}$};
\draw [color=ffqqqq](-1.53947,0.22428444444444524) node[anchor=north west] {$\mathbf{-}$};
\draw [color=ffqqqq](-1.5394666666666645,-0.119448888888887572) node[anchor=north west] {$\mathbf{-}$};
\draw [color=qqqqff](-0.2925866666666638,-1.2297955555555531) node[anchor=north west] {$\mathbf{-}$};
\draw [color=ffqqqq](1.5120533333333375,-0.14476444444444524) node[anchor=north west] {$\mathbf{+}$};
\draw [color=qqqqff](0.1672,-1.2366222222222197) node[anchor=north west] {$\mathbf{-}$};
\draw [fill=uuuuuu] (1.,0.) circle (1.5pt);
\draw[color=uuuuuu] (1.1638933333333372,0.11430222222222082) node {$(1, 0)$};
\draw [fill=uuuuuu] (0.,1.) circle (1.5pt);
\draw[color=uuuuuu] (0.15354666666666986,1.1171911111111112) node {$(0, 1)$};
\draw [fill=uuuuuu] (-1.,0.) circle (1.5pt);
\draw[color=uuuuuu] (-0.8094933333333308,0.10001777777777895) node {$(-1, 0)$};
\draw [fill=uuuuuu] (0.,-1.) circle (1.5pt);
\draw[color=uuuuuu] (0.20402666666666988,-1.1336888888888862) node {$(0, -1)$};
\draw [fill=uuuuuu] (0.,0.) circle (1.5pt);
\draw [fill=uuuuuu] (0.5527662281876642,0.8333363648462626) circle (1.5pt);
\draw[color=uuuuuu] (0.98928,0.84257777777778) node {T($\cos\alpha$,$\sin\alpha$)};
\draw [fill=uuuuuu] (0.5527662281876642,0.) circle (1.5pt);
\draw[color=qqqqff] (0.704773333333337,0.4413511111111119) node {$\sin\alpha$};
\draw[color=ffqqqq] (0.3856533333333367,-0.07699555555555419) node {$\cos\alpha$};
\draw[color=qqwuqq] (0.26181333333333658,0.09319111111111231) node {$\alpha$};
\end{tikzpicture}

\

Funkciji tangens in kotangens prav tako definiramo s pomočjo enotske krožnice in poltraka iz izhodišča. Za ponazoritev $\tg\alpha$ narišemo vzporednico $y$-osi skozi $(1,0)$. V dobljenem pravokotnem trikotniku meri kateta, ki leži na $x$-osi, 1. Ordinata presečišča vzporednice in poltraka predstavlja velikost funkcije $\tg\alpha$. Za ponazoritev $\ctg\alpha$ pa narišemo vzporednico $x$-osi skozi (0,1). Abscisa presečišča vzporednice in poltraka predstavlja velikost funkcije $\ctg\alpha$. 

\definecolor{wqwqwq}{rgb}{0.3764705882352941,0.3764705882352941,0.3764705882352941}
\definecolor{qqwuqq}{rgb}{0.,0.39215686274509803,0.}
\definecolor{qqffqq}{rgb}{0.,1.,0.}
\definecolor{ffxfqq}{rgb}{1.,0.4980392156862745,0.}
\definecolor{cqcqcq}{rgb}{0.7529411764705882,0.7529411764705882,0.7529411764705882}
\definecolor{yqyqyq}{rgb}{0.5019607843137255,0.5019607843137255,0.5019607843137255}
\definecolor{uuuuuu}{rgb}{0.26666666666666666,0.26666666666666666,0.26666666666666666}
\begin{tikzpicture}[line cap=round,line join=round,>=triangle 45,x=3.0cm,y=3.0cm]
\draw[->,color=black] (-1.6944,0.) -- (1.9408,0.);
\foreach \x in {-1.5,-1.,-0.5,0.5,1.,1.5}
\draw[shift={(\x,0)},color=black] (0pt,-2pt);
\draw[color=black] (1.8384,0.0256) node [anchor=south west] { x};
\draw[->,color=black] (0.,-1.1556) -- (0.,1.6476);
\foreach \y in {-1.,-0.5,0.5,1.,1.5}
\draw[shift={(0,\y)},color=black] (-2pt,0pt);
\draw[color=black] (0.032,1.5068) node [anchor=west] { y};
\clip(-1.6944,-1.1556) rectangle (1.9408,1.6476);
\draw [shift={(0.,0.)},line width=2.pt,color=qqwuqq,fill=qqwuqq,fill opacity=0.10000000149011612] (0,0) -- (0.:0.192) arc (0.:55.093100573353155:0.192) -- cycle;
\draw [line width=1.6pt,color=wqwqwq] (0.,0.) circle (3.cm);
\draw [line width=1.pt,color=cqcqcq,domain=-1.6944:1.9408] plot(\x,{(-1.-0.*\x)/-1.});
\draw [line width=1.pt,color=cqcqcq] (1.,-1.1556) -- (1.,1.6476);
\draw [line width=2.pt,domain=0.0:1.9407999999999999] plot(\x,{(-0.--0.8200829734309459*\x)/0.5722446301090631});
\draw [line width=2.pt,color=ffxfqq] (0.,1.)-- (0.6977886977886979,1.);
\draw [line width=2.pt,color=qqffqq] (1.,0.)-- (1.,1.4330985915492955);
\draw [color=ffxfqq](-1.2144,-0.3812) node[anchor=north west] {$\mathbf{+}$};
\draw [color=ffxfqq](0.4496,1.4748) node[anchor=north west] {$\mathbf{+}$};
\draw [color=qqffqq](1.224,0.9308) node[anchor=north west] {$\mathbf{+}$};
\draw [color=qqffqq](-1.0416,-0.6116) node[anchor=north west] {$\mathbf{+}$};
\draw [color=ffxfqq](-1.2272,0.7028) node[anchor=north west] {$\mathbf{-}$};
\draw [color=ffxfqq](0.6992,-0.6884) node[anchor=north west] {$\mathbf{-}$};
\draw [color=qqffqq](-0.9776,0.9588) node[anchor=north west] {$\mathbf{-}$};
\draw [color=qqffqq](1.0192,-0.4452) node[anchor=north west] {$\mathbf{-}$};
\draw [fill=uuuuuu] (1.,0.) circle (1.5pt);
\draw[color=uuuuuu] (1.1708,0.0956) node {$(1, 0)$};
\draw [fill=uuuuuu] (0.,1.) circle (1.5pt);
\draw[color=uuuuuu] (-0.1708,1.1196) node {$(0, 1)$};
\draw [fill=uuuuuu] (0.6977886977886979,1.) circle (1.5pt);
\draw [fill=uuuuuu] (1.,1.4330985915492955) circle (1.5pt);
\draw[color=ffxfqq] (0.3824,1.1132) node {$\ctg\alpha$};
\draw[color=qqffqq] (1.1748,0.674) node {$\tg\alpha$};
\draw[color=qqwuqq] (0.2636,0.0764) node {$\alpha$};
\end{tikzpicture}

\

Vrednost funkcije sinus in kosinus se ne spremeni, če vrednosti kota prištejemo večkratnik kota $2\pi$. Zato sta periodični funkciji z osnovno periodo $2\pi$. Funkciji tangens in kotangens sta periodični z osnovno periodo $\pi$. Velja:
\begin{align*}
\sin\alpha&=\sin(\alpha+k 2\pi) \\
\cos\alpha&= \cos(\alpha +k2\pi) \\
\tg\alpha&= \tg(\alpha +k\pi)\\
\ctg\alpha&= \ctg(\alpha +k\pi); \quad k \in\mathbb{Z}
\end{align*}

Kote, ki so večji od $90^{\circ}$ in manjši od $360^{\circ}$, pretvarjamo v ostre kote po naslednjih obrazcih:
\begin{itemize}
\item $90^{\circ} <\alpha<180^{\circ}$ oz. $\frac{\pi}{2}<\alpha<\pi$
\begin{align*}
\sin(180^{\circ}-\alpha)&=\sin\alpha \\
\cos(180^{\circ}-\alpha)&=-\cos\alpha \\
\tg(180^{\circ}-\alpha) &=-\tg\alpha \\
\ctg(180^{\circ}-\alpha)&= -\ctg\alpha
\end{align*}

\item $180^{\circ} <\alpha<270^{\circ}$ oz. $\pi<\alpha<\frac{3\pi}{2}$
\begin{align*}
\sin(180^{\circ}+\alpha)&=-\sin\alpha \\
\cos(180^{\circ}+\alpha)&=-\cos\alpha \\
\tg(180^{\circ}+\alpha) &=\tg\alpha \\
\ctg(180^{\circ}+\alpha)&= \ctg\alpha
\end{align*}

\item $270^{\circ} <\alpha<360^{\circ}$ oz. $\frac{3\pi}{2}<\alpha<2\pi$
\begin{align*}
\sin(360^{\circ}-\alpha)&=-\sin\alpha \\
\cos(360^{\circ}-\alpha)&=\cos\alpha \\
\tg(360^{\circ}-\alpha) &=-\tg\alpha \\
\ctg(3600^{\circ}-\alpha)&=-\ctg\alpha
\end{align*}
\end{itemize}

\subsection{Adicijski izreki}
Adicijske izreke uporabljamo za izračun kotnih funkcij vsote in razlike kotov, pri poenostavljanju izrazov in računanju enačb.
\begin{align*}
\cos(\alpha+ \beta) &= \cos\alpha \cdot \cos\beta - \sin\alpha \cdot \sin\beta \\
\cos(\alpha - \beta) &= \cos\alpha \cdot \cos\beta + \sin\alpha \cdot \sin\beta \\
\sin(\alpha + \beta) &= \sin\alpha \cdot \cos\beta + \cos\alpha \cdot \sin\beta \\
\sin(\alpha - \beta) &= \sin\alpha \cdot \cos\beta - \cos\alpha \cdot \sin\beta \\
\tg(\alpha + \beta) &= \frac{\tg\alpha + \tg\beta}{1- \tg\alpha \cdot \tg\beta} \\
\tg(\alpha - \beta) &= \frac{\tg\alpha - \tg\beta}{1+ \tg\alpha \cdot \tg\beta}
\end{align*}
Iz zgornjih enačb sledijo enačbe kotnih funkcij dvojnega kota:
\begin{align*}
\cos(2\alpha) &= \cos^2\alpha- \sin^2\alpha\\
\sin(2\alpha) &= 2\sin\alpha \cos\alpha \\
\tg(2\alpha) &= \frac{2\tg\alpha}{1- \tg^2\alpha} \\
\end{align*}

\subsection{Trigonometrične enačbe}

Trigonometrične enačbe so enačbe, v katerih nastopa neznanka kot argument katere od kotnih funkcij. Začnimo s preprostimi enačbami, v katerih nastopa le ena kotna funkcija.

\begin{itemize}
\item $\sin x=a \mid |a|\le 1$

Rešitev enačbe so presečišča grafov $y=\sin x$ in $y=a$. Upoštevati moramo,da je $\sin x$ periodična funkcija s periodo $2\pi$ in da funcija doseže vrednost $a$ pri dveh kotih. Rešitve so tako:
\begin{align*}
x_1&= \arcsin a + 2k\pi, k\in \mathbb{Z} \\
x_2&= \pi -\arcsin a + 2k\pi, k\in \mathbb{Z}
\end{align*}

\item $\cos x=a \mid |a|\le 1$

Rešitve enačbe so presečišča grafov $y=\cos x$ in $y=a$. Upoštevati moramo periodičnost in dva različna kota, kjer ima funkcija kosinus vrednost $a$. Rešitve so tako:
\begin{align*}
x_1&= \arccos a + 2k\pi, k\in \mathbb{Z} \\
x_2&= -\arccos a + 2k\pi, k\in \mathbb{Z}
\end{align*}

\item $\tg x =a$

Rešitve enačbe so presečišča grafov $y=\tg x$ in $y=a$. Funkcija tangens ima periodo $\pi$. Rešitve so tako:
\begin{equation*}
x= \arctg a +k\pi, k \in \mathbb{Z}
\end{equation*}
\end{itemize}

Kadar v enačbi nastopajo različne kotne funkcije, poskušamo vse izraziti le z eno. Pomagamo si z zvezami med kotnimi funkcijami in adicijskimi izreki. Če je enačba sestavljena samo iz členov $\sin^2x$, $\cos^2x$, $\sin\cdot\cos$, $1$(vsi členi 2. stopnje), lahko enačbo delimo s $\cos^2x$ in rešujemo kvadratno enačbo za $\tg x$. Za lažje reševanje lahko to uvedemo kot novo neznanko. 

\subsection{Grafi kotnih funkcij}
Kot smo spoznali, so kotne funkcije nekaj osnovnega za razumevanja tako preproste kot osnovne geometrije. Prav tako so nekaj posebnega tudi grafi kotnih funkcij, katerih periodičnost je iz njih lepo razvidna. \\

Graf, ki predstavlja funkcijo $f(x) = sinx$, imenujemo sinusoida. Pri risanju vedno privzamemo, da je argument $x$ kot, izražen v radianih.

% graf sinusa v osnovni obliki
\definecolor{ffqqqq}{rgb}{1.,0.,0.}
\definecolor{qqwuqq}{rgb}{0.,0.6,0.}
\definecolor{cqcqcq}{rgb}{0.75,0.75,0.75}
\begin{tikzpicture}[line cap=round,line join=round,>=triangle 45,x=1.0cm,y=1.0cm]
\draw [color=cqcqcq,, xstep=1.6cm,ystep=1.0cm] (-3.5,-2.5) grid (7.5,2.5);
\draw[->,color=black] (-3.5,0.) -- (7.5,0.);
\draw[shift={(-3.14,0)},color=black] (0pt,2pt) -- (0pt,-2pt) node[below] {\footnotesize $-\pi$};
\draw[shift={(-1.57,0)},color=black] (0pt,2pt) -- (0pt,-2pt) node[below] {\footnotesize $-\pi/2$};
\draw[shift={(1.57,0)},color=black] (0pt,2pt) -- (0pt,-2pt) node[below] {\footnotesize $\pi/2$};
\draw[shift={(3.14,0)},color=black] (0pt,2pt) -- (0pt,-2pt) node[below] {\footnotesize $\pi$};
\draw[shift={(4.71,0)},color=black] (0pt,2pt) -- (0pt,-2pt) node[below] {\footnotesize $3\pi/2$};
\draw[shift={(6.28,0)},color=black] (0pt,2pt) -- (0pt,-2pt) node[below] {\footnotesize $2\pi$};
\draw[->,color=black] (0.,-2.5) -- (0.,2.5);
\foreach \y in {-2.,-1.,1.,2.}
\draw[shift={(0,\y)},color=black] (2pt,0pt) -- (-2pt,0pt) node[left] {\footnotesize $\y$};
\draw[color=black] (0pt,-10pt) node[right] {\footnotesize $0$};
\clip(-3.5,-2.5) rectangle (7.5,2.5);
\draw[line width=1.3pt,color=qqwuqq,smooth,samples=100,domain=-3.5:7.5] plot(\x,{sin(((\x))*180/pi)});
\draw [line width=1.pt,dash pattern=on 2pt off 2pt,color=ffqqqq,domain=-3.5:7.5] plot(\x,{(--1.5707963267948966-0.*\x)/1.5707963267948966});
\draw [line width=1.pt,dash pattern=on 2pt off 2pt,color=ffqqqq,domain=-3.5:7.5] plot(\x,{(-4.40-0.*\x)/4.40});
\end{tikzpicture}

Opazimo, da je funkcija sinus periodična funkcija s periodo $2\pi$, ter da je to liha funkcija, torej
\[
sin(-x) = -sin(x)
\]
Sinus je funkcija, ki je definirana na celi realni osi, zavzame pa le vrednosti na intervalu $[-1, 1]$. Kot je razvidno z grafa, so ničle funkcije vsi večkratniki števila $\pi$ in sicer 
\[
x = k\pi, \:   k\in \mathbb{Z}.
\]

Funkcija periodično zavzame tudi maksimume in minimume. V točkah, kjer je 
\[
x = \frac{\pi}{2} + 2k\pi \: ; \: k \in \mathbb{Z},
\]
doseže svoje maksimume ($y = 1$), v točkah pa, kjer je  
\[
x = \frac{3\pi}{2} + 2k\pi \: ; \: k \in \mathbb{Z},
\] 
doseže svoje minimume ($y = -1$).

Očitno je, da je funkcija sinus zvezna.\\

Funkcija $f(x) = cosx$ si je s sinusom zelo podobna.

% graf kosinusa v osnovni obliki
\definecolor{ffqqqq}{rgb}{1.,0.,0.}
\definecolor{qqwuqq}{rgb}{0.,0.6,0.}
\definecolor{cqcqcq}{rgb}{0.75,0.75,0.75}
\begin{tikzpicture}[line cap=round,line join=round,>=triangle 45,x=1.0cm,y=1.0cm]
\draw [color=cqcqcq,, xstep=1.57cm,ystep=1.0cm] (-3.5,-2.5) grid (7.5,2.5);
\draw[->,color=black] (-3.5,0.) -- (7.5,0.);
\draw[shift={(-3.14,0)},color=black] (0pt,2pt) -- (0pt,-2pt) node[below] {\footnotesize $-\pi$};
\draw[shift={(-1.57,0)},color=black] (0pt,2pt) -- (0pt,-2pt) node[below] {\footnotesize $-\pi/2$};
\draw[shift={(1.57,0)},color=black] (0pt,2pt) -- (0pt,-2pt) node[below] {\footnotesize $\pi/2$};
\draw[shift={(3.14,0)},color=black] (0pt,2pt) -- (0pt,-2pt) node[below] {\footnotesize $\pi$};
\draw[shift={(4.71,0)},color=black] (0pt,2pt) -- (0pt,-2pt) node[below] {\footnotesize $3\pi/2$};
\draw[shift={(6.28,0)},color=black] (0pt,2pt) -- (0pt,-2pt) node[below] {\footnotesize $2\pi$};
\draw[->,color=black] (0.,-2.5) -- (0.,2.5);
\foreach \y in {-2.,-1.,1.,2.}
\draw[shift={(0,\y)},color=black] (2pt,0pt) -- (-2pt,0pt) node[left] {\footnotesize $\y$};
\draw[color=black] (0pt,-10pt) node[right] {\footnotesize $0$};
\clip(-3.5,-2.5) rectangle (7.5,2.5);
\draw[line width=1.3pt,color=qqwuqq,smooth,samples=100,domain=-3.5:7.5] plot(\x,{cos(((\x))*180/pi)});
\draw [line width=1.pt,dash pattern=on 2pt off 2pt,color=ffqqqq,domain=-3.5:7.5] plot(\x,{(--6.283185307179586-0.*\x)/6.283185307179586});
\draw [line width=0.8pt,dash pattern=on 2pt off 2pt,color=ffqqqq,domain=-3.5:7.5] plot(\x,{(-3.141592653589793-0.*\x)/3.141592653589793});
\end{tikzpicture}

Razlikujeta se pravzaprav le v tem, da je, če opazujemo grafa funkcij, graf kosinusa za $\frac{\pi}{2}$ zamaknjen graf sinusa v levo. Tokrat je funkcija kosinus soda, kar pomeni, da velja
\[
cos(-x) = cosx
\]
Definicijsko območje in zaloga vrednosti sta enaka kot pri sinusu, se pa kosinus razlikuje v ničlah in ekstremih. Funkcija kosinus ima ničle v točkah 
\[
x = \frac{\pi}{2} + k\pi \: ; \: k \in \mathbb{Z},
\]
maksimume zavzame pri
\[
x =2\pi + 2 k\pi, \:   k\in \mathbb{Z}
\]
in minimume pri
\[
x =\pi + 2 k\pi, \:   k\in \mathbb{Z}.
\]

Funkcija $f(x) = tanx$ je, kot smo že prej spoznali, pravzaprav funkcija razmerja med sinusom in kosinusom:
\[
f(x) = tanx = \frac{sinx}{cosx}.
\]
Tangens torej zavzame vrednost $f(x) = 0$ v ničlah sinusa, v ničlah kosinusa pa ima graf tangensa pole. 
Definicijsko območje funkcije je torej unija intervalov ($\frac{\pi}{2} + k\pi$, $\frac{3\pi}{2} + k\pi$), pri čemer je $k \in \mathbb{Z}$. \\

% graf tangensa v osnovni obliki
\definecolor{ffqqqq}{rgb}{1.,0.,0.}
\definecolor{qqwuqq}{rgb}{0.,0.6,0.}
\definecolor{cqcqcq}{rgb}{0.75,0.75,0.75}
\begin{tikzpicture}[line cap=round,line join=round,>=triangle 45,x=1.0cm,y=1.0cm]
\draw [color=cqcqcq,, xstep=1.57cm,ystep=1.0cm] (-4.5,-4.3) grid (4.5,4.3);
\draw[->,color=black] (-4.7,0.) -- (4.7,0.);
\draw[shift={(-3.14,0)},color=black] (0pt,2pt) -- (0pt,-2pt) node[below] {\footnotesize $-\pi$};
\draw[shift={(-1.57,0)},color=black] (0pt,2pt) -- (0pt,-2pt) node[below] {\footnotesize $-\pi/2$};
\draw[shift={(1.57,0)},color=black] (0pt,2pt) -- (0pt,-2pt) node[below] {\footnotesize $\pi/2$};
\draw[shift={(3.14,0)},color=black] (0pt,2pt) -- (0pt,-2pt) node[below] {\footnotesize $\pi$};
\draw[->,color=black] (0.,-4.6) -- (0.,4.6);
\foreach \y in {-4.,-3.,-2.,-1.,1.,2.,3.,4.}
\draw[shift={(0,\y)},color=black] (2pt,0pt) -- (-2pt,0pt) node[left] {\footnotesize $\y$};
\draw[color=black] (0pt,-10pt) node[right] {\footnotesize $0$};
\clip(-4.5,-4.3) rectangle (4.5,4.3);
\draw[line width=1.3pt,color=qqwuqq,smooth,samples=100,domain=-4.5:-1.58] plot(\x,{tan(((\x))*180/pi)});
\draw[line width=1.3pt,color=qqwuqq,smooth,samples=100,domain=-1.56:1.56] plot(\x,{tan(((\x))*180/pi)});
\draw[line width=1.3pt,color=qqwuqq,smooth,samples=100,domain=1.58:4.5] plot(\x,{tan(((\x))*180/pi)});
\draw [line width=1.pt,dash pattern=on 2pt off 2pt,color=ffqqqq] (-1.57,-4.3) -- (-1.57,4.3);
\draw [line width=1.pt,dash pattern=on 2pt off 2pt,color=ffqqqq] (1.57,-4.3) -- (1.57,4.3);
\end{tikzpicture}

Od kotnih funkcij nam ostane le še funkcija $f(x) = cotx$, ki pa je pravzaprav inverz funkcije tangensa. Ničle ima tam, kjer ima tangens pole, in pole tam, kjer ima tangens pole.
Graf funkcije je torej takšen: \\

% graf kotangensa v osnovni obliki
\definecolor{ffqqqq}{rgb}{1.,0.,0.}
\definecolor{qqwuqq}{rgb}{0.,0.6,0.}
\definecolor{cqcqcq}{rgb}{0.75,0.75,0.75}
\begin{tikzpicture}[line cap=round,line join=round,>=triangle 45,x=1.0cm,y=1.0cm]
\draw [color=cqcqcq,, xstep=1.57cm,ystep=1.0cm] (-4.5,-4.3) grid (4.5,4.3);
\draw[->,color=black] (-4.7,0.) -- (4.7,0.);
\draw[shift={(-3.14,0)},color=black] (0pt,2pt) -- (0pt,-2pt) node[below] {\footnotesize $-\pi$};
\draw[shift={(-1.57,0)},color=black] (0pt,2pt) -- (0pt,-2pt) node[below] {\footnotesize $-\pi/2$};
\draw[shift={(1.57,0)},color=black] (0pt,2pt) -- (0pt,-2pt) node[below] {\footnotesize $\pi/2$};
\draw[shift={(3.14,0)},color=black] (0pt,2pt) -- (0pt,-2pt) node[below] {\footnotesize $\pi$};
\draw[->,color=black] (0.,-4.6) -- (0.,4.6);
\foreach \y in {-4.,-3.,-2.,-1.,1.,2.,3.,4.}
\draw[shift={(0,\y)},color=black] (2pt,0pt) -- (-2pt,0pt) node[left] {\footnotesize $\y$};
\draw[color=black] (0pt,-10pt) node[right] {\footnotesize $0$};
\clip(-4.5,-4.3) rectangle (4.5,4.3);
\draw[line width=1.3pt,color=qqwuqq,smooth,samples=100,domain=-4.5:-3.15] plot(\x,{cot(((\x))*180/pi)});
\draw[line width=1.3pt,color=qqwuqq,smooth,samples=100,domain=-3.13:-0.01] plot(\x,{cot(((\x))*180/pi)});
\draw[line width=1.3pt,color=qqwuqq,smooth,samples=100,domain=0.01:3.13] plot(\x,{cot(((\x))*180/pi)});
\draw[line width=1.3pt,color=qqwuqq,smooth,samples=100,domain=3.15:4.5] plot(\x,{cot(((\x))*180/pi)});
\draw [line width=1.2pt,dash pattern=on 2pt off 2pt,color=ffqqqq] (-3.14,-4.3) -- (-3.14,4.3);
\draw [line width=1.2pt,dash pattern=on 2pt off 2pt,color=ffqqqq] (0.,-4.3) -- (0.,4.3);
\draw [line width=1.2pt,dash pattern=on 2pt off 2pt,color=ffqqqq] (3.14,-4.3) -- (3.14,4.3);
\end{tikzpicture}


\section{Vaje}
\label{sec:sin-cos-vaje}

%%%%%%%%%%%%%%%%%%%%%%%%%%%%%%%%%%%%%%%%%%%%%%%%%%%%%%%%%%%%%%%%%%%%%%
% Odpremo datoteko, v katero se bodo zapisali odgovori za
% to poglavje.

% Določimo ime datoteke, v katero se bodo pisali odgovori.
% Vsako poglavje mora imeti svojo datoteko.
\def\datotekaOdgovori{odgovori-sincos}

% Odpremo datoteko z odgovori.
\Opensolutionfile{odgovor}[\datotekaOdgovori]

%%%%%%%%%%%%%%%%%%%%%%%%%%%%%%%%%%%%%%%%%%%%%%%%%%%%%%%%%%%%%%%%%%%%%%
% VAJE
%
% Sem vstavimo vaje s pomočjo okolja "vaja". Odgovor napišemo v vajo,
% v okolje "odgovor".

\begin{vaja}
V pravokotnem trikotniku, kjer je kot $\beta= 90^{\circ}$, merita stranici $a=5$ in $b=13$. Z uporabo kotnih funkcij določi kota $\alpha$ in $\gamma$ na minuto natančno, ter izračunaj stranico $c$. Izračunaj še vrednosti $\sin\gamma$, $\cos\alpha$, $\tg\alpha$ in $\ctg\alpha$. Kote izračunaj na minuto natančno, stranico na dve mesti natančno, vrednosti kotnih funkcij pa na štiri decimalna mesta natančno.
	\begin{odgovor}
$\alpha=22^{\circ}37'$, $\gamma=67^{\circ}23'$, $c=12$, $\sin\gamma=\cos\alpha=0.9231$, $\tg\alpha=0.4167$ in $\ctg\alpha=2.4000$
	\end{odgovor}
\end{vaja}

\begin{vaja}
V enkokrakem trapezu merijo stranice $a=21$, $c=13$ in krak $b=7$. Izračunaj kot, ki ga oklepata stranica $a$ in $b$ na tri veljavna mesta natančno.
	\begin{odgovor}
$\alpha=55,2^{\circ}$
	\end{odgovor}
\end{vaja}

\begin{vaja}
V rombu merita diagonali $e=26,8$ in $f=12,5$. Izračunajte ostri kot romba na minuto natančno.
	\begin{odgovor}
$\alpha=50^{\circ}1'$
	\end{odgovor}
\end{vaja}

\begin{vaja}
  Narišite graf funkcije, ki je podan s predpisom $f(x) = 3sin(4x)$. Kje ima funkcija ničle, maksimume in minimume, kakšna sta definicijsko območje in zaloga vrednosti?

  \begin{odgovor}
Graf funkcije je takšen:\\
	\definecolor{qqwuqq}{rgb}{0.,0.6,0.}
	\definecolor{cqcqcq}{rgb}{0.75,0.75,0.75}
    \begin{tikzpicture}[line cap=round,line join=round,>=triangle 45,x=1.0cm,y=1.0cm]

	\draw [color=cqcqcq,, xstep=1.57cm,ystep=1.0cm] (-4.3,-4.3) grid (6.,4.3);
	\draw[->,color=black] (-4.3,0.) -- (6.,0.);
	\draw[shift={(-3.141592653589793,0)},color=black] (0pt,2pt) -- (0pt,-2pt) node[below] {\footnotesize $-\pi$};
	\draw[shift={(-1.5707963267948966,0)},color=black] (0pt,2pt) -- (0pt,-2pt) node[below] {\footnotesize $-\pi/2$};
	\draw[shift={(1.5707963267948966,0)},color=black] (0pt,2pt) -- (0pt,-2pt) node[below] {\footnotesize $\pi/2$};
	\draw[shift={(3.141592653589793,0)},color=black] (0pt,2pt) -- (0pt,-2pt) node[below] {\footnotesize $\pi$};
	\draw[shift={(4.71238898038469,0)},color=black] (0pt,2pt) -- (0pt,-2pt) node[below] {\footnotesize $3\pi/2$};
	
	\draw[->,color=black] (0.,-4.3) -- (0.,4.3);
	\foreach \y in {-4.,-3.,-2.,-1.,1.,2.,3.,4.}
	\draw[shift={(0,\y)},color=black] (2pt,0pt) -- (-2pt,0pt) node[left] {\footnotesize $\y$};
	\draw[color=black] (0pt,-10pt) node[right] {\footnotesize $0$};
	\clip(-4.3,-4.3) rectangle (6.,4.3);
	\draw[line width=1.5pt,color=qqwuqq,smooth,samples=100,domain=-4.3:6] plot(\x,{3.0*sin((4.0*(\x))*180/pi)});


\end{tikzpicture}
  \end{odgovor}
\end{vaja}

\begin{vaja}
  Podana je funkcija $f(x) = \frac{1}{2} cos(2x)$. Zapišite vsa presečišča grafa funkcije z osjo $x$ in z osjo $y$. Koordinate naj bodo izračunane točno. Zapišite tudi definicijsko območje in zalogo vrednosti. 

  \begin{odgovor}
    Rešitev bi bila tu.
  \end{odgovor}
\end{vaja}

%%%%%%%%%%%%%%%%%%%%%%%%%%%%%%%%%%%%%%%%%%%%%%%%%%%%%%%%%%%%%%%%%%%%%%
% Treba je zapredi datoteko z odgovori

\Closesolutionfile{odgovor}

%%%%%%%%%%%%%%%%%%%%%%%%%%%%%%%%%%%%%%%%%%%%%%%%%%%%%%%%%%%%%%%%%%%%%%
% Odgovori

\section{Odgovori}
\label{sec:sincos-odgovori}

% Vključimo odgovore.
\input{\datotekaOdgovori}


%%% Local Variables:
%%% mode: latex
%%% TeX-master: "vaje"
%%% End:



\end{document}
